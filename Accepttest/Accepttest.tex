\documentclass[11pt]{article}
\usepackage[utf8]{inputenc}  % To control and create table of content
\usepackage{fancyhdr} 	% To create header
\usepackage{dirtytalk} % To create citations
\usepackage{array} % To control and create fixed size tables
\usepackage{longtable}
\usepackage{graphicx}
\usepackage{enumitem}
\usepackage{multirow} %Merging across columns and rows
\graphicspath{ {Illustrationer/} }
\usepackage{tikz}

\pagestyle{fancy}
\fancyhf{}
\lhead{Kravspecifikation}
\rhead{Version 0.1}
\rfoot{Page \thepage}

\renewcommand*\contentsname{Indholdsfortegnelse}
\newcommand{\namesigdatehrule}[1]{\par\tikz \draw [black, densely dotted, ultra thick] (0,0) -- (#1,0);\par}
\newcommand{\namesigdate}[2][3cm]{%
	\begin{minipage}{#1}%
		#2 \vspace{0.5cm}\namesigdatehrule{#1}\smallskip
		\small \noindent\textit{Underskrift}
		\vspace{0.5cm}\namesigdatehrule{#1}\smallskip
		\small \textit{Dato}
	\end{minipage}
}

\begin{document}
	\begin{titlepage}
		\begin{center}
			\Large\textbf{Accepttest}\\
			\large\textit{Version: 0.1} \linebreak \vspace{3cm}
		
			\noindent \namesigdate{Vejleder: Peter Johansen} \hfill \namesigdate[3cm]{Karl-Johan Schmidt}
			\hfill \namesigdate[3cm]{Simon Vammen Grønbæk}
			\linebreak \vspace{1cm} \linebreak
			
		\end{center}
	\end{titlepage}
	

	
	\tableofcontents
	\newpage
	
	\section{Funktionelle krav}
	
	\subsection{Use case 1}
		\begin{center}
			\begin{longtable}{|p{1.5cm}|p{2cm}|p{3cm}|p{3cm}|p{1.1cm}|}
			\hline
			Krav nr.: & Handling & Forventet resultat & Testmetode & Resul-tat  \\\hline
		    1.1.1& Medicinsk personale placerer manchetten på patienten & Manchetten sidder tæt om armen, så trykket fordeles ligeligt over hele området. Hele velcro hæfte siden skal fæstnes i filtsiden. & Manchetten trækkes løst over armen og fastspændes så den er placeret tætsiddende omkring overarmen med 2-3 cms afstand fra albuehulen  &   \\\hline
			1.1.2& Knappen [Start/Stop] trykkes & \multirow{3}{3cm}{Der vises et patient ID på skærmen} &\multirow{3}{3cm}{Knappen [Start/Stop] trykkes}  & \multirow{3}{3cm}{}  \\ \cline{1-2}
			1.1.3& Et nyt patient ID genereres & &  &  \\ \cline{1-2}
			1.1.4& Patient ID’et vises på skærmen & &  &   \\ \hline
			1.1.5& Blodtrykket måles via use case 3 & \multicolumn{3}{l|}{Se krav nr. 1.3.1 til 1.3.5} \\ \hline
			1.1.6& Manchetten fyldes med luft til et tryk på 25 mmHG over systolisk tryk (minimum 180 mmHg) & Manchet-trykket er 25 mmHg over det systoliske tryk & Aflæs tryk på analogt barometer. Systolisk tryk - manchettryk = 25 mmHg &  \\ \hline
			1.1.7& Tidsstempel gemmes når trykket er opnået & Tidsstemplet er gemt i loggen & Tjekke tidsstempling på SD kortet &   \\ \hline
		
			1.1.8& Trykket opretholdes i 5 minutter (Okklusion) og resterende tid vises på displayet & Manchet trykket holdes på mindst systolisk tryk + 10 mmHg i 5 min & Observere analogt barometer i 5 min &  \\ \hline
			1.1.9& Blodtrykket måles via use case 3 fra punkt 2. & \multicolumn{3}{l|}{Se krav nr. 1.3.2 til 1.3.5}  \\ \hline
			1.1.10& Deflaterer cuffen helt og forbliver i dette stadie i 5 min (Reperfusion). Ved deflation start gemmes tidsstempel. Tid til næste okklusion vises på displayet & Manchet trykket er < 10 mmHg i 5 min. kontinuerlig tids nedtælling vises på display. Tidsstempel for deflation start kan aflæses på fil.  & Observér analogt barometer 5 min. Tjekke tidsstempling på SD kortet &  \\ \hline
			1.1.11& Gentag punkt 7-11 (en cyklus) fire gange. Det nuværende cyklus nummer vises i displayet & Det specificerede antal cyklusser gennemfører & Observér at det totale antal cyklusser er tilsvarende antallet vist på displayet (tallet til højre for cyklus nr.) &  \\ \hline 
		\end{longtable}
		\end{center}
		
		\subsection*{Extension}
		\begin{center}
			\begin{longtable}{|p{1.5cm}|p{2cm}|p{3cm}|p{3cm}|p{1.1cm}|}
				\hline
				Krav nr.: & Handling & Forventet resultat & Testmetode & Resul-tat  \\\hline
				1.1.ex1 & Et patient ID eksisterer allerede på apparatet. Der genereres ikke noget nyt patient ID & Allerede eksisterende logfil vedføjes data. Ingen ny logfil generes og det gamle ID vises på skærmen & Kør use case 1 to gange og observér antallet af logfiler, samt ID på display er det samme hver gang &  \\
				\hline
				1.1.ex2 & Blodtrykket kunne ikke måles. Gentag use case 3 hvis extension 2 ikke lige er eksekveret. Ellers skrives i display “FEJL kunne ikke måle blodtryk” og use casen stopper  & Ved gentagne fejl ved blodtryksmåling skrives en fejlmeddelelse i displayet & Montér manchet på cylinder og start use case 1. observere antal oppustning. Efter 2 opfyldninger af manchetten observeres displayet & \\ \hline
			\end{longtable}
		\end{center}

	\subsection{Use case 2}
	\begin{center}
		\begin{longtable}{|p{1.5cm}|p{2cm}|p{3cm}|p{3cm}|p{1.1cm}|}
			\hline
			Krav nr.: & Handling & Forventet resultat & Testmetode & Resul-tat  \\\hline
			1.2.1 & Brugeren trykker på [Mål blodtryk] & \multirow{3}{3cm}{Der vises et patient ID på skærmen} & \multirow{3}{3cm}{Knappen [Start/Stop] trykkes} & \multirow{3}{3cm}{}  \\ \cline{1-2} 
			1.2.2 & Et nyt patient ID genereres & & &  \\ \cline{1-2}
			1.2.3 & Patient ID’et vises på skærmen & & &  \\ \hline
			1.2.4 & & \multicolumn{3}{l|}{Se krav nr. 1.3.1 til 1.3.5} \\ 
			\hline
			
		\end{longtable}
	\end{center}
	
	\subsection*{Extension}
	\begin{center}
		\begin{longtable}{|p{1.5cm}|p{2cm}|p{3cm}|p{3cm}|p{1.1cm}|}
			\hline
			Krav nr.: & Handling & Forventet resultat & Testmetode & Resul-tat  \\\hline
			1.2.ex1 & Et patient ID eksisterer allerede på apparatet. Der genereres ikke noget nyt patient ID & Allerede eksisterende logfil vedføjes data. Ingen ny logfil generes og det gamle ID vises på skærmen & Kør use case 2 to gange og observér antallet af logfiler, samt ID på display er det samme hver gang &  \\ \hline
			1.2.ex2 & Blodtrykket kunne ikke måles. Gentag use case 3 hvis extension 2 ikke lige er eksekveret. Ellers skrives i display “FEJL kunne ikke måle blodtryk” og use casen stopper.  & Ved gentagne fejl ved blodtryksmåling skrives en fejlmeddelelse i displayet & Montér manchet på cylinder og start use case 2. observere antal oppustning. Efter to opfyldninger af manchetten observeres displayet & \\ \hline
		\end{longtable}
	\end{center}
	
	\pagebreak
			\subsection{Use case 3}
			\begin{center}
				\begin{longtable}{|p{1.5cm}|p{2cm}|p{3cm}|p{3cm}|p{1.1cm}|}
					\hline
					Krav nr.: & Handling & Forventet resultat & Testmetode & Resul-tat  \\\hline
					1.3.1 & Manchetten fyldes til tryk over systoliske niveau & & &  \\ \hline
					1.3.2 & Luften lukkes gradvist ud og det systoliske tryk måles & \multirow{3}{3cm}{Trykket stemmer overens med reference apparatet med en tolerance på: Mean error +/- 5mmHg. Se EN 1060-3, punkt 7.9} & \multirow{3}{3cm}{Det målte tryk sammenlignes med trykket målt fra S-105B} & \multirow{3}{3cm}{}  \\ \cline{1-2} 
					1.3.3 & Middel blodtrykket måles & & & \\ \cline{1-2} 
					1.3.4 &  Det diastoliske tryk udregnes ud fra MAP og systoliske tryk  & & & \\ \hline
					1.3.5 & Blodtrykket vises på displayet og værdien gemmes i hukommelse & Systolisk, diastolisk og MAP vises på displayet & Gennemfør testmetode 1.1.1 til 1.1.5 & \\ \hline
				\end{longtable}
			\end{center}
			
			\pagebreak
					\subsection{Use case 4}
					\begin{center}
						\begin{longtable}{|p{1.5cm}|p{2cm}|p{3cm}|p{3cm}|p{1.1cm}|}
							\hline
							Krav nr.: & Handling & Forventet resultat & Testmetode & Resul-tat  \\\hline
							1.4.1 & Tag SD kortet ud af blodtryksapparatet  & \multirow{3}{3cm}{Der eksisterer en logfil på SD-kortet og den kopieres til det lokale drev}& \multirow{3}{3cm}{Indsæt SD-kort i computer. Tjek om fil eksisterer på kortet. Overfør fil til computeren.}& \multirow{3}{3cm}{} \\ \cline{1-2}
							1.4.2 & Sæt SD kortet i computeren og overfør filen & & & \\ \hline
							1.4.3 & Formatér SD kortet & \multirow{3}{3cm}{SD-kortet er formateret og tomt for filer} & \multirow{3}{3cm}{Formatér SD til FAT32. Indsæt SD-kort i apparatet og foretag blodtryksmåling. Tjek om logfil oprettes på SD-kort. } & \multirow{3}{3cm}{}\\ [2cm]\cline{1-2}
							1.4.4 & Sæt SD kortet tilbage i konditioneringsapparatet & & & \\ \hline
						\end{longtable}
					\end{center}
					
					\subsection{Use case 5}
					\begin{center}
						\begin{longtable}{|p{1.5cm}|p{2cm}|p{3cm}|p{3cm}|p{1.1cm}|}
							\hline
							Krav nr.: & Handling & Forventet resultat & Testmetode & Resul-tat  \\\hline
							1.5.1 & Saturation og puls detekteres & Der kan aflæses en puls og saturation på displayet& Testes med reference pulsoximeter & \\ \hline
							1.5.2 & Saturation gemmes på SD-kort & Saturation et gemt på på SD-kort & Tjekke tidsstempling og saturation på SD kortet & \\ \hline
							\pagebreak \hline
							1.5.3 & Saturation er \textgreater90\%  & \multirow{2}{3cm}{Saturation er \textgreater90\% } & \multirow{2}{3cm}{Afklem arm i 5 min og test med reference pulsoximeter} & \multirow{2}{3cm}{}  \\ \cline{1-2}
							1.5.4 & Behandlingen kan fortsætte & & & \\ \hline
							\end{longtable}
					\end{center}
					
					\subsection*{Extensions}
					\begin{center}
						\begin{longtable}{|p{1.5cm}|p{2cm}|p{3cm}|p{3cm}|p{1.1cm}|}
							\hline
							Krav nr.: & Handling & Forventet resultat & Testmetode & Resul-tat  \\\hline
							1.5.ex1.1 & Tegn på dårlig kredsløb: Blodtryksapparatet stopper konditionerings forløbet & Saturation er \textless90\% og der gemmes et tidsstempel for hvor der afbrydes & Afklem arm til saturationen er  under niveau og observe displayet.
							Tjekke tidsstempling på SD kortet & \\ \hline
						\end{longtable}
				\end{center}
				
						\subsection{Use case 6}
						\begin{center}
							\begin{longtable}{|p{1.5cm}|p{2cm}|p{3cm}|p{3cm}|p{1.1cm}|}
								\hline
								Krav nr.: & Handling & Forventet resultat & Testmetode & Resul-tat  \\\hline
								1.6.1 & Montere manchetten på arm/ben & Manchetten sidder tæt om arm/ben. Hele velcro hæfte siden skal fæstnes i filtsiden & Manchetten trækkes løst over arm/ben og fastspændes så det ønskede område afklemmes. Kontrollér om manchetten er passer i størrelsen & \\ \hline
								1.6.2 & Tryk på knap [Start/Stop] & Luftpumpen startet & Knappen [Start/Stop] trykkes & \\ \hline
								1.6.3 & Manchetten pumpes op til 100mmHg & \multirow{2}{3cm}{Trykket i mancheten er 100 mmHg med en tolerance på +/- 10 mmHg}& \multirow{2}{3cm}{Observere analogt barometer i 3 min}& \multirow{2}{3cm}{} \\ \cline{1-2}
								1.6.4 & Trænings-sættet begyndes og trykkes holdes konstant på 100mmHg (+/-5mmHg) & & & \\ \hline
								1.6.5 & Tryk på knap [Start/Stop]  & \multirow{2}{3cm}{Manchet trykket er < 10 mmHg efter 1min.} & \multirow{2}{3cm}{Knappen [Start/Stop] trykkes og observer at trykket på det analog barometer} & \multirow{2}{3cm}{} \\ [2cm] \cline{1-2}
								1.6.6 & Manchetten deflateres  & & & \\ \hline
							\end{longtable}
						\end{center}
						
					
					\subsection{Use case 7}
					\begin{center}
						\begin{longtable}{|p{1.5cm}|p{2cm}|p{3cm}|p{3cm}|p{1.1cm}|}
							\hline
							Krav nr.: & Handling & Forventet resultat & Testmetode & Resul-tat  \\\hline
							1.7.1 & Brugeren trykker på knappen [Start/Stop] & \multirow{2}{3cm}{Use casen afbryges og manchetten tømmes for luft} & \multirow{2}{3cm}{Start use case 1, 3 eller 6 og på et vilkårligt tidspunkt tryk på knappen [Start/Stop]} & \multirow{2}{3cm}{} \\ \cline{1-2}
							1.7.2 & Den igangværende use case afbrydes & & & \\ \hline
							1.7.3 & Manchetten tømmes for luft og tidsstempel med “Gennemført afklemning = false” gemmes i hukommelsen & Manchet trykket er < 10 mmHg efter 1 min. Tidsstempel for  “Gennemført afklemning = false” gemmes i fil & Observere analogt barometer 1 min.
							Tjek tidsstempling på SD kortet & \\ \hline
						\end{longtable}
					\end{center}
					
					\subsection*{Extension}
					\begin{center}
						\begin{longtable}{|p{1.5cm}|p{2cm}|p{3cm}|p{3cm}|p{1.1cm}|}
							\hline
							Krav nr.: & Handling & Forventet resultat & Testmetode & Resul-tat  \\\hline
							1.7.ex1 & Use case 6 er aktiv: ingen data gemmes i hukommelsen & Der gemmes ingen data & Kør use case 6 efterfulgt af use case 7 og observér logfilen & \\ \hline
						\end{longtable}
					\end{center}
					
					\subsection{Use case 8}
					\begin{center}
						\begin{longtable}{|p{1.5cm}|p{2cm}|p{3cm}|p{3cm}|p{1.1cm}|}
							\hline
							Krav nr.: & Handling & Forventet resultat & Testmetode & Resul-tat  \\\hline
							1.8.1 & Brugeren trykker på knappen [Start/Stop] for at vælge Tid pr cyklus & Ved knaptryk på [Start/Stop] vælges “Tid pr cyklus” og det valgte område begynder at blinke & Tryk på knappen [Start/Stop] og observér displayet & \\ \hline
							1.8.2 & Bruger trykker på knappen [Mål blodtryk] for at ændre Tid pr cyklus & Værdien i det valgte område ændres med 30s per tryk. & Tryk på knappen [Mål blodtryk] og observér ændringen  & \\ \hline
							1.8.3 & Bruger trykker på knappen [Start/Stop] for at gemme ændringen & Værdien gemmes og det valgte område stopper med at blinke & Tryk på knappen [Start/Stop] og observér displayet. Start use case 1 og tjek occlutionstid & \\ \hline
							1.8.4 & Bruger trykker på knappen [Mål blodtryk] for at navigere til Antal cyklusser & Ved knaptryk på [Mål blodtryk] flyttes den firkantede markør på displayet til “Antal cyklusser” & Tryk på knappen [Mål blodtryk] og observér ændringen  & \\ \hline
							1.8.5 & Ved knap tryk på [Start/stop] vælges Antal cyklusser & Ved knaptryk på [Start/Stop] vælges “Antal cyklusser” og det valgte område begynder at blinke & Tryk på knappen [Start/Stop] og observér displayet & \\ \hline
							1.8.6 & Ved knap tryk på [Mål blodtryk] ændre Antal af cyklusser & Værdien i det valgte område ændres med 1 per tryk. & Tryk på knappen [Mål blodtryk] og observér ændringen & \\ \hline
							1.8.7 & Brugeren trykker på knappen [Start/Stop] for at gemme ændringen & Værdien gemmes og det valgte område stopper med at blinke & Tryk på knappen [Mål blodtryk] og observér ændringen.
							start use case 1 og tjek total antal cyklusser & \\ \hline
						\end{longtable}
					\end{center}
					
		
		
	\section{Ikke funktionelle krav}
			\subsection{Microcontroller}
			\begin{center}
				\begin{longtable}{|p{1.5cm}|p{2cm}|p{3cm}|p{3cm}|p{1.1cm}|}
					\hline
					Krav nr.: & Handling & Forventet resultat & Testmetode & Resul-tat  \\\hline
					 2.1.1 & Type: Atmega32 & Atmega32 & Visuel inspektion af microcontroller & \\ \hline
				\end{longtable}
			\end{center}
			
			\subsection{Filformat og opsætning}
			\begin{center}
				\begin{longtable}{|p{1.5cm}|p{2cm}|p{3cm}|p{3cm}|p{1.1cm}|}
					\hline
					Krav nr.: & Handling & Forventet resultat & Testmetode & Resul-tat  \\\hline
					2.2.1 & Data logged i formatet .csv og hver kolonne indeholder følgende værdier og enheder:  & \multirow{7}{3cm}{Logfil er kommasepareret og at enhederne stemmer overens. of filen er af type .csv }&  \multirow{7}{3cm}{Inspicer logfil i texteditor (Gedit, notepad, textedit osv.)}& \multirow{7}{3cm}{} \\ \cline{1-2}
					2.2.1a& Tidsstempel [yy:mm:dd hh:mm:ss] & &  & \\ \cline{1-2}
					2.2.1b& Afklemnings-tryk [mmHg] & &  & \\ \cline{1-2}
					2.2.1c&  Gennemført afklemning [Boolean] & &  & \\ \cline{1-2}
					2.2.1d&  Systoliske blodtryk [mmHg] & &  & \\ \cline{1-2}
					2.2.1e&  Middel-blodtryk (MAP) [mmHg] & &  & \\ \cline{1-2}
					2.2.1f&  Diastolisk blodtryk [mmHg]  & &  & \\ \hline
					2.2.2 & Der oprettet én fil pr patient, med filnavn tilsvarende det unikke patient ID og apparatets ID i følgende format: “PatientID \_ApparatID”  & En enkel fil eksistere på SD-kortet. filnavnet består af “PatientID\_ApparatID” & Kør use case 2 flere gange og observer antallet og navngivningen af logfil(er)  & \\ \hline

				\end{longtable}
			\end{center}
			
			\pagebreak
			\subsection{Patient ID}
			\begin{center}
				\begin{longtable}{|p{1.5cm}|p{2cm}|p{3cm}|p{3cm}|p{1.1cm}|}
					\hline
					Krav nr.: & Handling & Forventet resultat & Testmetode & Resul-tat  \\\hline
					2.3.1 & Består af karaktererne A-Z og tallene 0-9 & PatientID består af karaktererne  & Kør use case 2  og observer navngivningen af logfil & \\ \hline
					2.3.1a& ID’et er fem karakterer lang: ***** svarende til 60 millioner kombinationer & \multirow{2}{3cm}{A-Z og tallene 0-9} & \multirow{2}{3cm}{Visuel inspektion af logfilen}  & \multirow{2}{3cm}{} \\ \cline{1-2}
					2.3.1b& ID’et er ikke case sensitiv  & &  & \\ \hline
				\end{longtable}
			\end{center}
			
			\subsection{Hukommelse}
			\begin{center}
				\begin{longtable}{|p{1.5cm}|p{2cm}|p{3cm}|p{3cm}|p{1.1cm}|}
					\hline
					Krav nr.: & Handling & Forventet resultat & Testmetode & Resul-tat  \\\hline
					2.4.1& Information lagres på micro SDSC af typen: & \multirow{3}{3cm}{SD kortet er af typen micro SDSC, class 4, fat32 og minimum 128mb} & \multirow{3}{3cm}{Tag SD kortet ud og se specifikationer}  & \multirow{3}{3cm}{}  \\ \cline{1-2}
					2.4.1a& Class 4 & &  & \\ \cline{1-2}
					2.4.1b& Fil system [fat32] og minimum 128mb  & &  & \\ \hline
				\end{longtable}
			\end{center}
			
			\pagebreak
			\subsection{Forsyning}
			\begin{center}
				\begin{longtable}{|p{1.5cm}|p{2cm}|p{3cm}|p{3cm}|p{1.1cm}|}
					\hline
					Krav nr.: & Handling & Forventet resultat & Testmetode & Resul-tat  \\\hline
					2.5.1& Konditione-ringsap-paratet skal forsynes med 12V, min 2A & &  & \\ \hline
					2.5.1a& DC-connector, ydre Ø=5,5mm, indre Ø = 2,1  & Connectoren har målene: ydre Ø=5,5mm, indre Ø = 2,1 mm & Mål med skydelære  & \\ \hline
					2.5.1b& 8 stk AAA batterier (1,5V)  & 8 stk AAA batterier & Visuel inspektion  & \\ \hline
				\end{longtable}
			\end{center}
			
			\subsection{Fysiske krav}
			\begin{center}
				\begin{longtable}{|p{1.5cm}|p{2cm}|p{3cm}|p{3cm}|p{1.1cm}|}
					\hline
					Krav nr.: & Handling & Forventet resultat & Testmetode & Resul-tat  \\\hline
					2.6.1& Knapper & \multirow{4}{3cm}{Knapperne er tilstede på apparatet}  & \multirow{4}{3cm}{Visuel inspektion}  & \multirow{4}{3cm}{}  \\ \cline{1-2}
					2.6.1a& [Start/Stop] & &  & \\ \cline{1-2}
					2.6.1b& [Mål blodtryk] & &  & \\ \cline{1-2}
					2.6.2& Hvert apparat udstyres med et unik serie nummer, kaldet apparat ID & &  & \\ \hline
				\end{longtable}
			\end{center}
	
	\end{document}