	\chapter{Indledning}
	Dokumentet specificere de krav der eksisterer til produktet, Konditioneringsapparatet, som har til formål at udføre konditionerings behandling på patienter der har fået en apopleksi. Apparatet udviklings i samarbejde med læge Rolf Blauenfeldt, Neurologisk afsnit, Aarhus Universitetshospital (AUH). Med udgangspunkt i et blodtryksapparat skal produktet kunne måle blodtrykket og derefter skabe arteriel okklusion i en specificeret tidsperiode, efterfulgt af en pause. Dette gentages i et fastsat antal cyklusser. Produktet indgår i et forskningsprojekt, hvor patienten med AIS som udgangspunkt skal behandles med RIPC så snart de møder præhospitalet, og apparatet og behandling forsættes under og efter indskrivelse på hospitalet. 
	Kravene til Konditioneringsapparatet er fastsat over flere faser. Som udgangspunkt er kravene blevet specificeret af projektets ophavsmand, Rolf Blauenfeldt. Dette er sket igennem flere møder og mailkorrespondancer i projektets opstartsfase. Kravene omkring produktet også skal kunne håndtere okklusionstræning er kommet til i samarbejde med vejleder, Peter Johansen. 
	
	\section{Formål}
	Kravspecifikation udarbejdes for at sikre enighed mellem kunden, vejleder og projektgruppen inden udviklingsfasen igangsættes. Dokumentet beskriver samtlig funktionaliteter for Konditioneringsapparatet. Udviklingen af prototypen og hele projektet er en iterativ proces og derfor kan krav opdateres undervej. 
	
	\section{Læsevejledning og dokumentstruktur}
	Dokumentet er struktureret således at første beskrives det overordnede system som helhed. Dernæst beskrives de funktionelle krav til produktet. De funktionelle krav er struktureret som fully dressed use cases og beskriver forskellige krav som scenarier. De ikke funktionelle krav afgrænser projektet, og er beskrevet i punktform. Til sidst beskrives user interfacet. 
		\newpage
	\section{Versionshistorik}
		\begin{longtable}{ |p{0.24\textwidth} |p{0.42\textwidth}| p{0.25\textwidth}|  } 
			\hline
			\rowcolor{usDef}
			\textbf{Versions nummer} &  \textbf{Ændring} & \textbf{Dato og initialer} \\
			\hline
			0.1 &  Oprettelse af dokument & 12.09.15 KJS \\
			\hline
			0.2 & Splittet use cases ud i seperate .tex filer og tilføjelse af \textit{samarbejdspartner}& 25.09.15 KJS \\
			\hline
			0.3 & Rettelser efter review & 27.09.15 KJS \\
			\hline
			0.4 &  Tilføjelse af use case diagram & 30.10.15 SVG \\
			\hline
			0.5 &  Tilføjelse af system oversigt & 07.10.15 KJS, SVG \\
			\hline
			0.6 &  Rettelser efter samtale med Rolf& 23.11.15 KJS, SVG \\
			\hline
			0.7 &  Rettelser efter system test& 26.11.15 KJS, SVG \\
			\hline
		\end{longtable}

	
	\section{Definitioner og forkortelser}
	\begin{longtable}{ |p{0.34\textwidth} |p{0.6\textwidth}| } 
		\hline
		\textbf{Udtryk / Forkortelse} &  \textbf{Forklaring} \\
		\hline
		RIPC & Remote ischemic pre/per/post conditioning. Længerevarende okklusion af ydre ekstremitet, efterfulgt af en deflations fase\\
		\hline
		AIS / apopleksi & Acute ischemic stroke, en pludseligt opstået neurologisk skade eller udfald på baggrund af iskæmi (nedsat blodforsyning) i hjernen \\
		\hline
		AUH & Aarhus Universitetshospital \\
		\hline
		\textit{Konditioneringsapparatet} & Navnet på prototype som udvikles til at udføre RIPC \\
		\hline
		\textit{Okklusionsfase} & Periode hvor på manchetten skaber arteriel okklusion \\
		\hline
		\textit{Deflationsfase} & Periode der er altid er efterfulgt en okklusionsfase, hvor manchetten er deflateret i under 50mmHg\\
		\hline
		\textit{Cyklus} & Forløb bestående af én \textit{okklusionfase} og én \textit{deflationsfase} \\
		\hline
		\textit{Gennemført afklemning} & Boolean værdi der bruges til at bestemme om en cyklus er gennemfør eller ej \\
		\hline
		\textit{Tid pr cyklus} & Værdi, der angiver hvor mange sekunder en cyklus skal være. For at simplificere use cases er denne værdi fastsat til 5 minutter, men det kan ændres \\
		\hline
		\textit{Antal cyklusser} & Værdi, der angiver hvor mange cyklusser konditioneringen skal vare. For at simplificere use cases er denne værdi fastsat til 4 cyklusser \\
		\hline
	\end{longtable}
	
	\section{Baggrund}
	Beskrivelse af projektet: 
	{\textit{Akut blodprop i hjernen (Acute Ischemic Stroke – AIS) er en førende årsag til død og alvorlig handicap hos personer over 60 år.Intravenøs trombolysebehandling administreret indenfor 4,5 time fra symptomdebut er den nuværende bedste medicinske behandling. Grundet sikkerhedshensyn og det snævre tidsvindue er det desværre kun et fåtal af AIS patienterne, der modtager denne behandling. Målet er at opløse blodproppen og genoprette blodforsyning og dermed redde hjernevæv, der lider af iltmangel men endnu ikke er dødt. Om et område af hjernen dør eller står til at redde ved en blodprop afhænger ikke kun af selve blodproppen men også om hjernen er i stand til at få blod via omveje dannet af hjernens små blodkar. Et område af hjernen går til grunde med det samme (infarktkernen). Denne kerne af dødt hjernevæv kan i dagene efter en blodprop sprede sig og vokse. Der er således behov for at kunne beskytte hjernen mod iltmangel og øge andelen af hjernevæv, der overlever en blodprop. Iltmangel induceret periodevis i et fjernt organ (remote ischemic conditioning RIC) kan udføres ved at puste en blodtryksmanchet med afklemning af armen. Konditionering kan leveres som pre, per, og postconditionering, afhængig af om stimulus udøves før iltmangel, under iltmangel men før blodproppen er opløst og endelig efter blodproppen er opløst. Dyrestudier og senest kliniske studier har vist at RIC kan mindske det område af hjertet eller hjernen, der dør ved en blodprop. Det er ikke tilstrækkeligt undersøgt om RIC mindsker risikoen for handicap efter en blodprop i hjernen.}
		
	\section{Samarbejdspartner}
	Bachelor gruppen kunde er Rolf Blauenfeldt, Neurologisk afsnit, Aarhus Universitetshospital (AUH). Projektet er udbudt af Rolf og det er i samarbejde med ham at projektet er blevet specificeret 
	
	Peter Johansen er vejleder for bachelor gruppen. Der afholdes faste møder med vejlederen hver anden uge, hvor der gives det status over projektet og diskuteres aktuelle problemstilling
	
	Anders Esager og Anders Toft er bachelor gruppen review partner. De har også underskrevet tavshedserklæring og har derfor mulighed for at blive sat ind i projektet og fungere som sparringspartnere. 
	