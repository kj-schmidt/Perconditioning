	\subsection{Use case 6 - Okklusionstræning} \label{title:uc6}
	Her beskrives hvordan prototypen skal udføre okklusionstræning og hvordan scenariet skal forløbe, se use case 6 i tabel \ref{tab:uc6}. Her er den primære aktør skiftet fra \textit{medicinsk personale} til \textit{bruger}.
	\begin{table}[H]
		\begin{center}
			\begin{tabular}{ | p{0.24\textwidth} | p{0.7\textwidth}| } 
				\hline
				Mål & Gennemføre okklusion af venøs kredsløb under træning  \\ 
				\hline
				Initiering &  Bruger\\
				\hline
				Aktører og interessenter & 
				\begin{itemize}
					\item Bruger (primær)
				\end{itemize} \\ 
				\hline
				Referencer & - \\ 
				\hline
				Antal samtlige forekomster & En pr træningspas \\ 
				\hline	
				Startbetingelser & 
				\begin{itemize}
					\item Mode switch er sat til  “\textit{okklusionstræning}”
 				\end{itemize} \\ 
				\hline
				Slutbetingelser & 
				\begin{itemize}
					\item Okklusions træningssæt gennemført
				\end{itemize} \\ 
				\hline
				Normal forløb & \begin{enumerate}
					\setlength\itemsep{0cm} % Decrease line distance
					\item Montere manchetten på arm/ben
					\item Tryk på knap [Start/Stop]
					\item Manchetten pumpes op til 100mmHg
					\item Træningssættet begyndes og trykket holdes konstant på 100mmHg (+/-10mmHg)
					\item Efter træningssættet trykker \textit{bruger} stop
					\item Manchetten deflateres
				\end{enumerate} \\ 
				\hline
				Undtagelser & - \\ 
				\hline
			\end{tabular}
		\end{center}
		
			\caption{\textit{Fully dressed} use case diagram over use case 6} \label{tab:uc6}
		\end{table}
	\newpage
		