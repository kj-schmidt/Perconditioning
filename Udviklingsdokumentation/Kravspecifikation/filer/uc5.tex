	\subsection{Use case 5 - Sikkerhedskontrol med pulsoximeter}
	Denne use case viser hvordan \textit{Konditioneringsapparatet} skal håndtere sikkerhedskontrol under et konditioneringsforløb, (Se tabel \ref{tab:uc5}).
	\begin{table}[H]
		\begin{center}
			\begin{tabular}{ | p{0.24\textwidth} | p{0.7\textwidth}| } 
				\hline
				Mål & Sikre at patientens kredsløb tåler konditionering \\ 
				\hline
				Initiering &  Konditionering (UC1)\\
				\hline
				Aktører og interessenter & 
				\begin{itemize}
					\item Patient (sekundær)
				\end{itemize} \\ 
				\hline
				Referencer & - \\ 
				\hline
				Antal samtlige forekomster & Ingen \\ 
				\hline	
				Startbetingelser & 
				\begin{itemize}
					\item Konditionering (UC1) (Se \ref{title:uc1}) igangværende
					\item Pulsoximeteret er monteret på patients finger
					\item Patient har gennemført én afklemnings cyklus
 				\end{itemize} \\ 
				\hline
				Slutbetingelser & 
				\begin{itemize}
					\item Patients tilstand er bestemt 
				\end{itemize} \\ 
				\hline
				Normal forløb & \begin{enumerate}
					\setlength\itemsep{0cm} % Decrease line distance
					\item Saturation detekteres
					\item Saturation gemmes på SD-kort
					\item Saturation er tilfredsstillende
					\subitem [Undtagelse \#1.1][Undtagelse \#1.2]
					\item Behandlingen kan fortsætte
				\end{enumerate} \\ 
				\hline
				Undtagelser &  [Undtagelse \#1.1] Tegn på dårlig kredsløb: Blodtryksapparatet stopper konditionerings forløbet 
				[Undtagelse \#1.2] Kør use case 7\\ 
				\hline
			\end{tabular}
		\end{center}
		\caption{\textit{Fully dressed} use case diagram over use case 5} \label{tab:uc5}
	\end{table}
			\newpage