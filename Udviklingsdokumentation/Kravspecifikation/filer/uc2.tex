	\subsection{Use case 2 - Initialiser blodtryksmåling }
	Denne use case beskriver hvordan en blodtryksmåling initieres. Use casen er gældende når en blodtryksmåling skal foretages uden af apparatet udfører konditioneringsbehandling, (Se tabel \ref{tab:uc2}).
	\begin{table}[H]
		\begin{center}
			\begin{tabular}{ | p{0.24\textwidth} | p{0.7\textwidth}| } 
				\hline
				Mål& Mål et blodtryk\\ 
				\hline
				Initiering &  Medicinsk personale\\
				\hline
				Aktører og interessenter & 
				\begin{itemize}
					\item Medicinsk personale(primær)
					\item Patient (sekundær)
				\end{itemize} \\ 
				\hline
				Referencer & Mål blodtryk(UC3) \ref{title:uc3} \\ 
				\hline
				Antal samtlige forekomster & Ingen\\ 
				\hline	
				Startbetingelser & 
				\begin{itemize}
					\item Manchetten er placeret på armen
					\item Mode switch er sat til “\textit{Konditionering}”
				\end{itemize} \\ 
				\hline
				Slutbetingelser & 
				\begin{itemize}
					\item Patientens blodtryk er målt
				\end{itemize} \\ 
				\hline
				Normal forløb & \begin{enumerate}
					\setlength\itemsep{0cm} % Decrease line distance
					\item Brugeren trykker på [Mål blodtryk]
					\subitem [Undtagelse \#1]
					\item Et nyt patient ID genereres
					\subitem [Undtagelse \#2] 
					\item Patient ID’et vises på skærmen
					\item Blodtrykket måles via \textit{use case 3} (Se \ref{title:uc3})
					
				\end{enumerate} \\ 
				\hline
				Undtagelser & [Undtagelse \#1] SD kortet er ikke monteret korrekt
				
				[Undtagelse \#2] Et patient ID eksisterer allerede på apparatet. Der genereres ikke noget nyt patient ID. \\ 
				\hline
				
			\end{tabular}
		\end{center}
		\caption{\textit{Fully dressed} use case diagram over use case 2} \label{tab:uc2}
			\end{table}
		\newpage