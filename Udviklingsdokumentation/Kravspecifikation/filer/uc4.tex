	\subsection{Use case 4 - Overfør data}
	I tabel \ref{tab:uc4} beskrives forløbet når data skal eksporteres fra \textit{Konditioneringsapparatet}. Denne use case sikre at SD kortet håndteres ens hver gang data eksportes og dette mindsker fejl. 
	\begin{table}[H]
		\begin{center}
			\begin{tabular}{ | p{0.24\textwidth} | p{0.7\textwidth}| } 
				\hline
				Mål & Eksportér data fra blodtryksapparat til databasen\\ 
				\hline
				Initiering &  Medicinsk personale\\
				\hline
				Aktører og interessenter & 
				\begin{itemize}
					\item Medicinsk personale(primær)
					\item Patient (sekundær)
				\end{itemize} \\ 
				\hline
				Referencer & - \\ 
				\hline
				Antal samtlige forekomster & Én pr behandlingsforløb \\ 
				\hline	
				Startbetingelser & 
				\begin{itemize}
					\item Der eksisterer en logfil på hukommelsen
				\end{itemize} \\ 
				\hline
				Slutbetingelser & 
				\begin{itemize}
					\item Logfilen er overført til database
					\item Blodtryksapparat udstyres med formateret hukommelse og klar til næste patient
				\end{itemize} \\ 
				\hline
				Normal forløb & \begin{enumerate}
					\setlength\itemsep{0cm} % Decrease line distance
					\item Tag SD kortet ud af blodtryksapparatet 
					\item Sæt SD kortet i computeren og overfør filen 
					\item Formatér SD kortet
					\item Sæt SD kortet tilbage i \textit{konditioneringsapparatet} 
				\end{enumerate} \\ 
				\hline
				Undtagelser & -\\ 
				\hline
			\end{tabular}
		\end{center}
		
			\caption{\textit{Fully dressed} use case diagram over use case 4} \label{tab:uc4}
		\end{table}
			\newpage