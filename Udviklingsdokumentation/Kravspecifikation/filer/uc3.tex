	\subsection{Use case 3 - Mål blodtryk}
	Uce case 3 kan ses i fully dressed form på tabel \ref{tab:uc3}. Denne use case beskriver forløbet under en blodtryksmåling.
	\begin{table}[H]
		\begin{center}
			\begin{tabular}{ | p{0.24\textwidth} | p{0.7\textwidth}| } 
				\hline
				Mål & Mål et systolisk, diastolisk og middel(MAP) tryk\\ 
				\hline
				Initiering &  Konditionering (UC1) eller Initialiser blodtryksmåling (UC2)\\
				\hline
				Aktører og interessenter & - \\
				\hline
				Referencer & - \\ 
				\hline
				Antal samtlige forekomster & En til mange\\ 
				\hline	
				Startbetingelser & 
				\begin{itemize}
					\item Patient ID er oprettet
					\item Manchetten er placeret på armen
					\item Mode switch er sat til “\textit{Konditionering}”
				\end{itemize} \\ 
				\hline
				Slutbetingelser & 
				\begin{itemize}
					\item Blodtrykket er mål
				\end{itemize} \\ 
				\hline
				Normal forløb & \begin{enumerate}
					\setlength\itemsep{0cm} % Decrease line distance
					\item Manchetten fyldes til tryk over systolisk niveau 
					\item Luften lukkes gradvist ud og det systoliske tryk registreres 
					\item Middel trykket (MAP) måles 
					\item Det diastoliske tryk udregnes ud fra systole og MAP 
					\item Blodtrykket vises på skærmen og værdien gemmes i hukommelsen med et tidsstempel 
				\end{enumerate} \\ 
				\hline
				Undtagelser & -\\ 
				\hline
			\end{tabular}
		\end{center}
			\caption{\textit{Fully dressed} use case diagram over use case 3} \label{tab:uc3}
			\end{table}
			\newpage