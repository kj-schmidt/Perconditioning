	\section{Use case 8}
						\begin{longtable}{|p{0.1\linewidth}|p{0.2\linewidth}|p{0.2\linewidth}|p{0.2\linewidth}|p{0.1\linewidth}|}
							\hline
							\rowcolor{usDef}
							Krav nr.: & Handling & Forventet resultat & Testmetode & Resul-tat  \\\hline
							2.8.1 & Brugeren trykker på knappen [Start/Stop] for at vælge Tid pr cyklus & Ved knaptryk på [Start/Stop] vælges “Tid pr cyklus” og det valgte område begynder at blinke & Tryk på knappen [Start/Stop] og observér displayet & \\ \hline
							2.8.2 & Bruger trykker på knappen [Mål blodtryk] for at ændre Tid pr cyklus & Værdien i det valgte område ændres med 30s per tryk. & Tryk på knappen [Mål blodtryk] og observér ændringen  & \\ \hline
							2.8.3 & Bruger trykker på knappen [Start/Stop] for at gemme ændringen & Værdien gemmes og det valgte område stopper med at blinke & Tryk på knappen [Start/Stop] og observér displayet. Start use case 1 og kontroller okklutionstid & \\ \hline
							2.8.4 & Bruger trykker på knappen [Mål blodtryk] for at navigere til Antal cyklusser & Ved knaptryk på [Mål blodtryk] flyttes den firkantede markør på displayet til “Antal cyklusser” & Tryk på knappen [Mål blodtryk] og observér ændringen  & \\ \hline
							2.8.5 & Ved knap tryk på [Start/stop] vælges Antal cyklusser & Ved knaptryk på [Start/Stop] vælges “Antal cyklusser” og det valgte område begynder at blinke & Tryk på knappen [Start/Stop] og observér displayet & \\ \hline
							2.8.6 & Ved knap tryk på [Mål blodtryk] ændre Antal af cyklusser & Værdien i det valgte område ændres med 1 per tryk. & Tryk på knappen [Mål blodtryk] og observér ændringen & \\ \hline
							2.8.7 & \textit{Bruger} trykker på knappen [Start/Stop] for at gemme ændringen & Værdien gemmes og det valgte område stopper med at blinke & Tryk på knappen [Mål blodtryk] og observér ændringen.
							start use case 1 og tjek total antal cyklusser & \\ \hline
						\end{longtable}		