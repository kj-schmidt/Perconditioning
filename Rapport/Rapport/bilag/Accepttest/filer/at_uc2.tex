\section{Use case 2}
		\begin{longtable}{|p{0.1\linewidth}|p{0.2\linewidth}|p{0.2\linewidth}|p{0.2\linewidth}|p{0.1\linewidth}|}
			\hline
			Krav nr.: & Handling & Forventet resultat & Testmetode & Resul-tat  \\\hline
			2.2.1 & Brugeren trykker på [Mål blodtryk] & \multirow{3}{\linewidth}{Der vises et patient ID på skærmen} & \multirow{3}{\linewidth}{Knappen [Start/Stop] trykkes} & \multirow{3}{\linewidth}{}  \\ \cline{1-2} 
			2.2.2 & Et nyt patient ID genereres & & &  \\ \cline{1-2}
			2.2.3 & Patient ID’et vises på skærmen & & &  \\ \hline
			2.2.4 & & \multicolumn{3}{l|}{Se krav nr. 1.3.1 til 1.3.5} \\ 
			\hline
			
		\end{longtable}
	
	\section*{Extension}
		\begin{longtable}{|p{0.1\linewidth}|p{0.2\linewidth}|p{0.2\linewidth}|p{0.2\linewidth}|p{0.1\linewidth}|}
			\hline
			Krav nr.: & Handling & Forventet resultat & Testmetode & Resul-tat  \\\hline
			2.2.ex1 & Et patient ID eksisterer allerede på apparatet. Der genereres ikke noget nyt patient ID & Allerede eksisterende logfil vedføjes data. Ingen ny logfil generes og det gamle ID vises på skærmen & Kør use case 2 to gange og observér antallet af logfiler, samt ID på display er det samme hver gang &  \\ \hline
			2.2.ex2 & Blodtrykket kunne ikke måles. Gentag use case 3 hvis extension 2 ikke lige er eksekveret. Ellers skrives i display “FEJL kunne ikke måle blodtryk” og use casen stopper.  & Ved gentagne fejl ved blodtryksmåling skrives en fejlmeddelelse i displayet & Montér manchet på cylinder og start use case 2. observere antal oppustning. Efter to opfyldninger af manchetten observeres displayet & \\ \hline
		\end{longtable}