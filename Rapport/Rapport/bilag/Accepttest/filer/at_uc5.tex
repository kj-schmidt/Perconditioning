\section{Use case 5 - Sikkerhedskontrol med pulsoximeter}
					\begin{longtable}{|p{0.1\linewidth}|p{0.2\linewidth}|p{0.2\linewidth}|p{0.2\linewidth}|p{0.1\linewidth}|}
							\hline
							\rowcolor{usDef}
							Krav nr.: & Handling & Forventet resultat & Testmetode & Resul-tat  \\\hline
							2.5.1 & Saturation detekteres & Der kan aflæses en saturation på displayet& Testes med reference pulsoximeter \fxnote{Tilføj reference} & \\ \hline
							2.5.2 & Saturation gemmes på SD-kort & Saturation et gemt på SD-kortet & Kontroller tidsstempling og saturation på SD kortet & \\ \hline
							2.5.3 & Saturation er \textgreater90\%  & \multirow{2}{\linewidth}{Saturation er \textgreater90\% } & \multirow{2}{\linewidth}{Afklem arm i 5 min og test med reference pulsoximeter} & \multirow{2}{\linewidth}{}  \\ \cline{1-2}
							2.5.4 & Behandlingen kan fortsætte & & & \\ \hline
							\caption{Accepttest forløb for use case 5}
					\end{longtable}
					
					\newpage
					\section*{Undtagelser}
					\begin{longtable}{|p{0.1\linewidth}|p{0.2\linewidth}|p{0.2\linewidth}|p{0.2\linewidth}|p{0.1\linewidth}|}
						\hline
						\rowcolor{usDef}
						Krav nr.: & Handling & Forventet resultat & Testmetode & Resul-tat  \\\hline
						2.5.ex1.1 & Tegn på dårlig kredsløb: Blodtryksapparatet stopper konditionerings forløbet & Saturation er \textless90\% og der gemmes et tidsstempel for hvor der afbrydes & Afklem arm til saturationen er  under niveau og observe displayet.
						Kontroller tidsstempling på SD kortet & \\ \hline
						2.5.ex1.2 & Kør use case 7 & \fxnote{TILFØJ TESTS }& & \\ \hline
						\caption{Undtagelser for use case 5}
					\end{longtable}