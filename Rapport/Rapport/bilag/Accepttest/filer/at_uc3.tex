	\pagebreak
	\section{Use case 3 - Mål blodtryk}
				\begin{longtable}{|p{0.1\linewidth}|p{0.2\linewidth}|p{0.2\linewidth}|p{0.2\linewidth}|p{0.1\linewidth}|}
					\hline
					\rowcolor{usDef}
					Krav nr.: & Handling & Forventet resultat & Testmetode & Resul-tat  \\\hline
					2.3.1 & Manchetten fyldes til tryk over systoliske niveau &  \multirow{4}{\linewidth}{Trykket stemmer overens med reference apparatet med en tolerance på: Mean error +/- 5mmHg. Se EN 1060-3, punkt 7.9} &\multirow{4}{\linewidth}{Det målte tryk sammenlignes med trykket målt fra \textit{S-105B}} & \multirow{4}{\linewidth}{}  \\ \cline{1-2}
					2.3.2 & Luften lukkes gradvist ud og det systoliske tryk måles &  &  &   \\ \cline{1-2} 
					2.3.3 & Middel blodtrykket måles & & & \\ \cline{1-2} 
					2.3.4 &  Det diastoliske tryk udregnes ud fra MAP og systoliske tryk  & & & \\ \hline
					2.3.5 & Blodtrykket vises på displayet og værdien gemmes i hukommelsen med et tidsstempel & Systolisk, diastolisk og MAP vises på displayet & Gennemfør testmetode fra krav nr. 2.1.1 til 2.1.5 & \\ \hline
				\end{longtable}