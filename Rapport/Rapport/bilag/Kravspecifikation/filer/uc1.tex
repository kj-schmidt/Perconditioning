\subsection{Use case 1 - Konditionering}
\begin{center}
		\begin{longtable}{ | p{0.24\textwidth} | p{0.7\textwidth}| } 
			\hline
			Mål & Gennemføre én konditioneringsbehandling  \\ 
			\hline
			Initiering &  Medicinsk personale\\
			\hline
			Aktører og interessenter & 
			\begin{itemize}
				\item Medicinsk personale(primær)
				\item Patient (sekundær)
			\end{itemize} \\ 
			\hline
			Referencer & Use case 3 \\ 
			\hline
			Antal samtlige forekomster & En til mange\\ 
			\hline	
			Startbetingelser & 
			\begin{itemize}
				\item Mode switch er sat til “\textit{Konditionering}”
			\end{itemize} \\ 
			\hline
			Slutbetingelser & 
			\begin{itemize}
				\item \textit{Antal cyklusser} er gennemført og gemt på hukommelsen
			\end{itemize} \\ 
			\hline
			Normal forløb & \begin{enumerate}
				\setlength\itemsep{0cm} % Decrease line distance
				\item \textit{Medicinsk personale} placerer manchetten på patienten
				\item \textit{Medicinsk personale} trykker på knappen [Start/Stop]
				\subitem[Undtagelse \#1]
				\item Et nyt patient ID genereres
				\subitem[Undtagelse \#2] 
				\item Patient ID’et vises på skærmen
				\item Blodtrykket måles via \textit{use case 3}
				\item Blodtrykket vises på displayet og værdien gemmes i hukommelsen
				\item Manchetten fyldes med luft til et tryk på 25 mmHG over systolisk tryk (minimum 200 mmHg)					
				\item Tidsstempel gemmes når systoliske tryk er opnået
				\item Trykket opretholdes i 5 minutter(Okklusion) og resterende tid vises på displayet
				\item Deflaterer manchetten helt og forbliver i dette stadie i 5 min(Reperfusion) Ved deflation start gemmes tidsstempel. Tid til næste okklusion vises på displayet
				\item Gentag punkt 7-10 (en \textit{cyklus}) fire gange. Det nuværende cyklus nummer vises i displayet
			\end{enumerate} \\ 
			\hline
			Undtagelser & [Undtagelse \#1] SD kortet er ikke monteret korrekt
			
			[Undtagelse \#2] Et patient ID eksisterer allerede på apparatet. Der genereres ikke noget nyt patient ID. \\
			\hline
			\caption{\textit{Fully dressed} use case diagram over use case 1}
		\end{longtable}
		
	\end{center}