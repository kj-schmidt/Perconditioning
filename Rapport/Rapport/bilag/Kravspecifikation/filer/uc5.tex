	\subsection{Use case 5 - Sikkerhedskontrol med pulsoximeter}
		\begin{center}
			\begin{tabular}{ | p{0.3\textwidth} | p{0.7\textwidth}| } 
				\hline
				Goal& Sikre at patientens kredsløb tåler konditionering \\ 
				\hline
				Initiation &  Use case 1\\
				\hline
				Actors and stakeholders & 
				\begin{itemize}
					\item Patient (sekundær)
					\item Medicinsk personale (sekundær)
				\end{itemize} \\ 
				\hline
				References & - \\ 
				\hline
				Number of concurrent occurrences & En til mange \\ 
				\hline	
				Precondition & 
				\begin{itemize}
					\item Use case 1 igangværende
					\item Pulsoximeteret er monteret på patients finger
					\item Patient har gennemført én afklemnings cyklus
					\item Mode switch er sat til “\textit{Konditionering}”
 				\end{itemize} \\ 
				\hline
				Postcondition & 
				\begin{itemize}
					\item Patients tilstand er bestemt 
				\end{itemize} \\ 
				\hline
				Main scenario & \begin{enumerate}
					\setlength\itemsep{0cm} % Decrease line distance
					\item Saturation detekteres
					\item Saturation gemmes på SD-kort
					\item Saturation er tilfredsstillende
					\subitem [Extension \#1.1][Extension \#1.2]
					\item Behandlingen kan fortsætte
				\end{enumerate} \\ 
				\hline
				Extensions &  [Extension \#1.1] Tegn på dårlig kredsløb: Blodtryksapparatet stopper konditionerings forløbet 
				[Extension \#1.2] Kør use case 7\\ 
				\hline
			\end{tabular}
		\end{center}
			\pagebreak