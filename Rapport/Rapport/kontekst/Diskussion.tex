\chapter{Diskussion}

\section{Oscillometrisk fikseret-ratio}
Den oscillometriske fikseret-ratio metode er brugt i vid udstrækning til non-invasive målinger af det systoliske og diastoliske blodtryk. Det er derfor ikke unormalt at apparatet beskrevet i denne rapport under afsnit \ref{Fikseret-ratio}, anvender fikseret-ratio fastsat ud fra empirisk data. Flere studier har også vist at denne metode har en høj nøjagtighed.\fixme{Theory of the Oscillometric Maximum and the Systolic and Diastolic Detection Ratios} Problemet med denne rigide fortolkning, at det systoliske og diastoliske blodtryk altid befinder sig samme procentsats fra middel arterie trykke, opstår ved individernes forskellighed.

Jiankun et al\fixme{Error Mechanisms of the Oscillometric Fixed-Ratio BloodPressure Measurement Method} opstiller en matematisk model for den oscillometriske metode medregnet arterie eftergivenheden og undersøger ud fra dette hvilke faktorer, som påvirker den fikserede-ratio og hvor stor en afvigelse, fra den sande værdi dette giver. Resultaterne af denne gennemgang er teoretiske afvigelser på op til 58 mmHg ved svær arterie stivhed. Efter som at stive arterier ofte er til stede ved  arteriosklerose er apopleksi patienter (også beskrevet i afsnit \ref{chap:Baggrund}) særlig udsatte for fejlmålinger med fikseret-ratio metoden. Den korte forklaring på dette problem er ændringer af manchet oscillotionernes kurve brede. Kurven som dannes af peak ampletuderne af oscillotionerne (se figur \ref{fig:OscillometriskMetode}) ændre karakter ved ændring af arterie stivheden. Dette illustreres bedst ved at afbillede data med normaliseret manchettryk oscillotioner over manchettrykket på arterier forskellig eftergivenhed. 

\begin{minipage}[t]{0.5\textwidth}
På figur \ref{fig:ErrorMechanismOfFixedRatio} er fejl mekanismen ved fikseret-ratio bestemt systolisk tryk (SP) og diastolisk tryk (DP) illustreret. Peak ampletuderne er normaliseret, hvilket tydeliggør ændringerne i kurve bredden, når arterie eftergivenheden ændres. 

Ved normale arterievægge passerer de empiriske ratio værdier godt, men efter som arteriet afviger fra det normale stiger fejl estimationen af SP og DP i takt med afvigelsen af eftergivenheden. Hvis karrene er stivere end normalt, resulterer det i en overestimation af det systoliske tryk og en underestimation af det diastoliske tryk. Overestimationen finder sted fordi den konstante ratio for det systoliske tryk (SP/MAP) nu befinder sig på et tidligere tidspunkt i tid, hvor manchet trykket er højere og derfor overestimerers SP. På samme måde som det systolisk tryk overestimeres, underestimeres det diastoliske tryk fordi den konstante ratio for det diastolisk tryk (SP/MAP) nu befinder sig på et senere tidspunkt i tid, hvor manchettrykket er lavere. Det samme scenarie gør sig gældende bare modsat, for en blodtryksmåling på arterier med en højere eftergivenhed end normalt. Ændringer i arterievæggens eftergivenhed påvirker ikke estimationen af MAP, som altid befinder sig med de største oscillationer i manchetten.

Anvendelse af oscillometrisk fikseret-ratio metoden, til at måle blodtryk på patienter med apopleksi kan være problematisk på grund af arterie stivheden, som giver anledning til fejl estimationer på op til 58 mmHg. Det bør derfor overvejes om andre metoder til at estimere det sys- og diastoliske tryk skal anvendes i stedet, for at sikre en højere nøjagtighed af blodtryksmålingerne.

Overestimationen af det systoliske blodtryk optræder ikke som fejlkilde ved konditioneringen, da her blot ønskes en total okklution af arterierne. Overestimationen giver ikke anledning til et for lavt afklemningstryk, som tillader blod til den afklemte ekstremitet. Fejlkonditionering på grund af blodtryksmålingen opstår kun ved underestimering af SYS.
\end{minipage}
\begin{minipage}[t]{0.5\textwidth}
	\begin{figure}[H]
		\centering
		\includegraphics[width=1\textwidth]{billeder/ErrorFixed-Ratio.pdf}
		\caption{Fejl mekanismen i fixed-ratio metoden ved ændringer af arterie stivheden. Pc er manchet tryk. DP er det diastoliske tryk og SP er det systoliske tryk}\label{fig:ErrorMechanismOfFixedRatio}
	\end{figure}
	\fixme{billede Ref: Error Mechanisms of the Oscillometric Fixed-Ratio BloodPressure Measurement Method}
\end{minipage}

\section{Medicinsk godkendelse} \label{title:medGodkendelse}
I dette afsnit udledes hvilke krav produktet skal opfylde, for at kunne anvendes i klinisk forsøg (se afsnit \ref{title:studieprotokold}) og til hjemmebehandling (Se afsnit \ref{title:Hjemmebehandling} omkring hjemmebehandling). Dette er relevant at diskuterer, fordi dette bør overvejes inden \textit{konditioneringsapparatet} kan tages i brug på patienter.

\subsection{Medicinsk udstyr}
Apparater, som skal bruges i medicinsk sammenhæng skal godkendes af sikkerhedsmæssige årsager, men kravene er forskellige alt efter hvilken sammenhæng apparates skal bruges og hvor farligt apparatet potentielt kan være (klassificering). \textit{Konditioneringsapparatet} er af kategorien \textit{medicinsk udstyr}, fordi dens anvendelse er beskrevet i direktivet om medicinsk udstyr (MDD 93/42/EEC) Artikel 1, 2.a:

\begin{quote}
	"Medicinsk udstyr: ethvert instrument, apparat, udstyr, software,
	materiale eller anden genstand anvendt alene eller i kombination,
	herunder software, som af fabrikanten er beregnet til specifik anvendelse
	til diagnostiske og/eller terapeutiske formål, og som hører med
	til korrekt brug heraf, og som af fabrikanten er beregnet til anvendelse
	på mennesker med henblik på:
	\begin{itemize}
		\item Diagnosticering, forebyggelse, overvågning, behandling eller
		lindring af sygdomme
		\item Diagnosticering, overvågning, behandling, lindring af eller
		kompensation for skader eller handicap
		\item Undersøgelse, udskiftning eller ændring af anatomien eller en
		fysiologisk proces
		\item Svangerskabsforebyggelse
	\end{itemize}
	
	og hvis forventede hovedvirkning i eller på det menneskelige
	legeme ikke fremkaldes ad farmakologisk, immunologisk eller metabolisk
	vej, men hvis virkning kan understøttes ad denne vej"
	
	Kilde:  \fixme{Kilde til derektivet 93/42/EEC}
\end{quote}

Konditioneringsapparatet opfylder første punkt fordi den skal forebygge celledød i penumbra. Ydermere opfyldes punkt to også, fordi konditioneringensapparatet måler blodtrykket af patienten og dermed hjælper til diagnosticering af patienten.

\subsubsection{Klassificering af medicinsk udstyr}
Klassificering af medicisk udstyr er krævet, fordi dokumentationskravene varierer her af. Jo højere klasse des farligere er apparatet klasseficeret. I følge "MEDICAL DEVICES: Guidance document - Classification of medical devices" er konditioneringsapparatet klasse IIa. Klassificeringen er bestemt ud fra regel 10 punkt 3, som beskriver elektriske apparater til måling af blodtrykket non-invasivst.

\subsection{Medicinsk udstyr til klinisk forsøg}
Den nationale myndighed i Danmark er sundhedsstyrelsen, som stiller krav til medicinsk udstyr. Ud over at de skal kontaktes og indformeres inden et klinisk forsøg med ikke CE godkendt udstyr, så stiller de også krav til dokumentationen af apparatet. Det krævede dokumentation er opfyldelse af relevant krav fra bekendtgørelsens \textit{væsentlige krav}.\fixme{ref: sundhedsstyrelsen  Introduktion til klinisk afprøvning af medicinsk udstyr} De væsentlige krav kan læses i MDD Bilag 1.

\subsection{Medicinsk udstyr til hjemmebrug}
Når apparatet skal avendes uden observation af medicinsk personale, skal det være CE godkendt. For at opnå denne godkendelse skal det bemyndigede organ (f.eks. Mermaid Medical) godkende klassificeringen af apparatet og dokumentationen som opfylder enten MDD bilag II, IV, V eller VI. 

\begin{quote}
	"Når fabrikanten har underskrevet EF-overensstemmelseserklæringen, kan udstyret CE-mærkes."
	
	Kilde: \fixme{ref: sundhedsstyrelsen  CE-Mærkning}
\end{quote}

Til opsummering må det konkluderes at det skal nøje overvejes om konditioneringsapparatet skal CE godkendes med det samme, for at kunne bruges til flere studier og forsøg, eller om apparatet blot skal godkendes til det ene studie beskrevet i studieprotokolden (se \ref{title:studieprotokold}).

\section{Okklusionstræning og trykregulering}
Som okklusionstræning er implementeret lige nu i \textit{Konditioneringsapparatet} findes der ingen nedregulering af trykket, hvis en pudselig trykstigning opstår, og derfor variere trykket i manchetten meget under træning. Ved okklusionstræning af fx. biceps brachii, vil en normal blodtryksmanchet, pga. dens størrelse, sidde meget i vejen under okklusionstræning. Desuden vil manchetten side oven på en stor del af biceps musklen, og dette vil betyde store variationer i trykket mellem den koncentriske og excentriske fase. Denne effekt kan reduceres ved at udskifte den brede manchet til en smallere manchet, som i stedet kun blev placeret proximalt for biceps brachii (Se figur \ref{fig:okklcuff}). En anden måde at imødekomme trykvariationer på, er ved at montere en "reserve"-tank under okklusionstræningen. Dette vil betyde en større volume i det pneumatiske system og mindske trykændringer ved kontraktion af musklen. 
\begin{figure}[H]
	\centering
	\includegraphics[trim={1.5cm 0 1.5cm 0}, clip, width=0.5\textwidth]{billeder/okklusionCuff.png}
	\caption{Placering af small manchet ved træning af biceps brachii}\label{fig:okklcuff}
\end{figure}
\fixme{ref: http://www.halsesf.com/cns-fatigue/  dato: 3-12-2015}

\section{Pulsoximetri}
Under resultat afsnittet \ref{title:pulsOxi} er der beskrevet en test, som bruger pulsoximeter for at tjekket kredsløbets status på armen under RIC behandling. Der blev kun foretaget én af sådan en test og den blev foretaget på person som er sund og rask. Derfor er testen ikke et gyldigt grundlag for at eksludere pulsoximetri som sikkerhedskontrol. Men teorien bag pulsoximeteri, som beskrives i projektafgrænsnings afsnittet (\ref{title:sikkerhedskontrol}), sætter tydelige begrænsninger for brugen af pulsoximetri som sikkerhedskontrol og testen underbygger kun denne påstand. 

\section{Projektplanlægning og gennemførsel}
Dette afsnit omhandler hvad der gik godt, mindre godt i projektet og tanker om dette.

Projektgruppens medlemmer Karl-Johan og Simon, har begge baggrund som sundhedsteknologi studerende, hvilket resulterede i nogle manglende egenskaber og erfaringer til projektet. Særligt under udvikling af \textit{Konditioneringsapparatets} software, som er skrevet i C++, oplevede projektgruppen problemer, fordi begge medlemmer først skulle lære sproget. Det ville have været oplagt at en af gruppens medlemmer havde fulgt et fag på Ingeniørhøjskolen Aarhus Universitet, for at sikre en god udvikling af softwaren. På trods af den manglende C++ erfaring i projektgruppen lykkedes det at opnå et godt resultat af softwaren. Softwaren kan læses i den vedlagte .ZIP fil.

Projektgruppens forholdsvise lille størrelse har også betydet at ikke alle former for projektstyringsværktøjer, kan betale sig tidsmæssigt. I metodeafsnittet (se afsnit \ref{title:scrum}) er det beskrevet hvordan gruppen halvvejs inde i projektet fravalgte anvendelsen af Pivotaltracker. På samme måde kan det også diskuteres om alt den skrevne udviklingsdokumentation er for meget. Projektgruppen har brugt rigtig meget tid på at dokumenterer udviklingen og planlægningen af projektet, selv om at gruppens størrelse kun er på to mand. Det har under hele projektet været mulig at få en mundtlig dokumentation på grund af at begge medlemmer havde den samme kontorplads og arbejdstider.
Udviklingsdokumentationen har dog aldrig været set som overflødig, bl.a. på grund af kundens interesse i et masseproduceret konditioneringsapparatet. Den store mængde udviklingsdokumentation sikre muligheden for fremtidig udvikling af \textit{konditioneringsapparatet}. 

Unit test og Integrations test (se V-model figur \ref{fig:vmodel}) blev i projektet gennemført med en yderst iterativ process. De enkelte elementer beskrevet i system design dokumentet (se System design i udviklingsdokumentation) er udviklet en af gangen og testet om det ønskede element virkede efter hensigten. Hvis det testede element ikke virkede efter hensigten, er det detaljerede design (se Implementeringsdokument i udviklingsdokumentation) af elementet blevet rettet og en ny implementering efterfulgt af unit test blevet gennemført. Et eksempel, på denne iterative process, er det analoge filter, som er blevet tilrettet gennem flere iterationer med implementering, for at opnå den ønskede effekt. Der er altså ikke kun blevet udført en Unit test, per element i \textit{konditioneringsapparatet}, men flere og derfor er alle tests ikke blevet dokumenteret. De vigtigste og endelige Unit tests er dog dokumenteret i Implementeringsdokumentet. Integrations test, som er sammensætningen af to eller flere elementer, er på samme måde som Unit test sket over flere forsøg og igen er kun de vigtigste praktiske erfaringer beskrevet i Implementeringsdokumentet.

Under system design fasen blev der anvendt \textit{Modelio} (se afsnit \ref{title:udviklingsvaerktoejer}) til SysML diagrammer. Der var måde fordele og ulemper forbundet med det valg. \textit{Modelio} er ikke et tegneprogram, men et open source modellerings miljø, hvilket resulterer i en meget høj indlæringskurve. Projektgruppen har brugt store mængder af tid på at forstå \textit{Modelio}, som kunne have været sparet ved at anvende et allerede kendt program, så som \textit{Visio}. Værktøjet \textit{Modelio} giver til gengæld muligheden for kompilering af alle tegnede diagrammer, hvilket stærkt mindsker mængden af fejl. Programmet sikre desuden at de udarbejdede diagrammer overholder SysML standarderne. Samtidig er programmet open source, hvilket tillader projektgruppen af anvende \textit{Modelio} i fremtiden uden betaling til licenser.

Beslutningen om anvendelse af Git til versionsstyring har været rigtig god. Efter at projektgruppen havde lært at anvende git, er store mængde af tid sparet på versionsstyring af projektdokumenter og software.

