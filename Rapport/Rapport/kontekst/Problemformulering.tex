\chapter{Problemformulering}\label{title:problemformulering}
Som beskrevet i baggrundsafsnittet (Se afsnit \ref{chap:Baggrund}) ønsker en forsker gruppe ved Aarhus Universitet Hospital at undersøge effekten ved per og postkonditionering. Til dette formål skal der bruges et modificeret blodtryksapparat, som kan indgå i forskningsprojekt til at foretage per og postkonditionering på forsøgspersonerne. Kunden har i samarbejde med Aarhus Universitet udarbejdet et bachelor projekt opslag med følgende punkter:
\begin{itemize}
	\item Samarbejde med en dansk producent af blodtryksappart
	\item Samarbejde med forsøgsansvarlige læger omkring produktkrav
	\item Designe et modificeret blodtryksapparat
	\item Samarbejde med produktionsvirksomhed i Kina omkring udvikling af prototype 
	\item Test af prototype udfra præspecificerede data
\end{itemize}

I samarbejde med projektvejleder Peter Johansen og projektudbyder Rolf Blauenfeldt har bachelorprojektet ændret karakter, fra at prototypen skulle fremstilling hos en kinesisk producent, til at bachelor gruppen selv fremstiller en \textit{proof of concept} prototype. Selvom bachelorgruppen selv udvikler prototypen ønskes det stadig fra kundens side at der bliver samarbejdet med den danske producent, for at sikre at prototypen ville ligge sig tæt op af deres blodtryksmålere.

For at produktet skal kunne bruges til konditioneringsbehandling skal det kunne måle et blodtryk, hvor efter der afklemmes i specificerede cyklusser. Afklemningstrykket skal være 25 mmHg over systolisk tryk for at sikre tilstrækkelig arteriel okklusion. De specificerede antal cyklusser fungere så forholdet mellem okklusion og reperfusion er en-til-en. 

Fra kundens side lyder endvidere et krav til perkonditioneringsprotokolen kan ændres, hvis forskningen viser bedre effekt ved en anden protokol. De ændringer der skal kunne foretages i protokollen er tiden en cyklus varer og antallet af cyklusser en konditioneringsbehandling skal have. 

Da patienten der skal modtage konditioneringsbehandling skal have armen afklemt i længerevarende perioder, er der fra kundens side stillet et krav omkring sikkerhedskontrol. Sikkerhedskontrollen stiller krav til at prototypen skal foretage et kredsløbstjek og vurdere om patienten kan risikere at tage skade af de iskæmiske tilstande den afklemte ekstremitet udsættes for under behandlingen..

Udover behovet for et apparat der kan udføre perkonditionering, er der efter foreslag fra vejleder Peter Johansen et ønske til at prototypen skal kunne bruges til okklusionstræning. Som en separat funktion skal prototype kunne skifte mellem konditioningsforløb og okklusionstræningsforløb. Ved okklusionstræning er kravet at trykket i manchetten holdes konstant på omkring 100mmHg. 

