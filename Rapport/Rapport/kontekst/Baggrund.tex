\chapter{Baggrund} \label{chap:Baggrund}

Apopleksi (pludseligt opstået fokale neurologiske symptomer) opstår af infarkt eller en blødning. Ved infarkt nedsættes eller afbrydes blodforsyningen i visse område af hjernen og dette medfører iltmangel i det ramte område. I 85\% af tilfælde er apopleksi forårsaget af infarkt og 15\% skyldes blødning \fixme{Reference til "Basis i sygdomslære, side 399-402}

Hvert år indlægges ca. 12.000 danskere i forbindelse med apopleksi og i den vestlige verden er apopleksi det tredjehyppigste dødsårsag.\fixme{Reference program apopleksi, side 14}. Af de personer der overlever et apopleksi tilfælde, lever næsten 50\% af dem med varige men og 25\% af dem har behov for andres hjælp ved daglige aktiviteter. \fixme{Refence til fakta om apopleksi http://www.hjernesagen.dk/om-hjerneskader/bloedning-eller-blodprop-i-hjernen/fakta-om-apopleksi } Det høje antal tilfælde årligt og de mange personer med varige men har store omkostninger for sundhedssektoren.  I 2001 kostede apopleksi sundhedsvæsnet 1,8 milliarder kroner. \fixme{Reference til trombolyse økonomi side 17}

Den nuværende behandling af apopleksi og dets følgevirkning sker i flere forskellige trin; forbyggende, akut behandling og rehabilitering. 

Meget af den forebyggende behandling af apopleksi ligger i livstilsændringen. Faktorer for udvikling af apopleksi er bl.a. hypertension, hjerte-kar sygdomme, arteriosklerose og forhøjet kolesterol. 

For at opnå størst effekt af akut behandling af apopleksi skal behandlingen helst ske inden for 5 timer efter tilfældet indtræf. Behandlingen består som regel af en scanning for afgøre om der er tale om en blodprop eller en blødning. Hvis der er tale om en blodprop, vil patienten modtage trombolysebehandling 

Afhængig af méngraden består rehabiliteringen af genoptræning i forskellige form. Menene af apopleksi kan være alt fra talebesvær til halvsidig lammelse og derfor afhænger genoptræning også deraf. \fixme{https://www.sundhed.dk/borger/sygdomme-a-aa/hjerte-og-blodkar/sygdomme/apopleksi/behandling-ved-apopleksi/}

\section{Konditionering}
Remote ischemic conditioning (RIC) er en behandlingsform som har vist sig at være effektivt til at beskytte kroppen mod iskæmiske tilstande. Konditionering er en endogen adaptiv proces, som beskytter kroppens organismer mod iltmangel, hvis det forekommer i små doser. Når konditionering skal bruges som behandlingsform skaber man konditioneringstilstanden ved at afklemme arm eller ben, heraf kaldet remote ischemic conditioning, til et tryk der er minimum 25 mmHg højere end det systoliske tryk. For at undgå at patienten tager skade af okklusion udføres den i cyklusser med en iskæmisk periode efterfulgt af en reperfusions periode. Antallet af cyklusser og den tid en cyklus varer variere meget i litteraturen, men som oftest udføres mellem 3-8 cyklusser og med en varighed på 3 til 15 minutter. \fixme{Reference til Hess, Hoda, Bhatia}
Behandlingen kan både foretages før, under og efter der opstår et akut iskæmisk stroke. Hvis konditioneringen foretages før, kaldes det remote ischemic preconditioning(RIPreC, under kaldes det perconditiong(RIPerC) og efter postconditioning(RIPostC). For at illutreres hvordan konditionerings behandlingen struktureres og udføres se figur \ref{fig:cycles}

\begin{figure}[H]
	\includegraphics[width = \textwidth]{billeder/PrePerPostKonditionering.png}
	\caption{Oversigt over mulige mekanismer der kan aktiverings under konditionering}\label{fig:cycles}
\end{figure}

\fixme{Billede til Hess, Hoda og Bhatia}
\subsection{Mekanismer}
Ligesom at der endnu ikke er evidence for at en bestemt tid pr cyklus og bestemt antal cyklusser, så hersker der også stor tvivl i hvilke mekanismer som er afgørende for at konditioneringen har effekt. Dog er vist en lang række mekanismer og endogene respons som opstår når kroppen udsættes for konditionering. Ved behandling med RIC øges det cerebrale flow ved hjælp af en række mekanisme, bla. er der målt et stigende niveau af nitrit og microRNA-144. I iltfattige område omdannes nitrit til nitrit oxid og dette medfører dilation af blodkarrene. Nitrit har ydermere en effekt på mitokondrierne. Mitokondrierne producere energi til cellen og uden den dør cellen. Nitrit øger mitokondriernes tolerance over for iltmangel. MicroRNA-144 har påvirker circulation og hvis dens effekt blokeret reduceres effekten af RIC. 
RIC har også vist at aktivere en række mekanismer i forbindelse med nervesystemet (Se figur \ref{fig:mechanism}. Under RIC er der en øget aktivitet af det autonome nervesystem, og især vagusnerven er aktiv. Blokeres responsen fra vagusnerven mindske effekt af RIC. Dette skyldes at vagusnerven indgår i et anti-inflammatoriske system, og øget aktivering af nerven reducerer inflammation ved fx. iskæmisk-reperfusionsskader. Overordnet set har studie vist at kroppens immun respons øges under RIC behandling. \fixme{Ref til Shimizu M, Saxena P, Konstantinov IE, Cherepanov V, Cheung MM, Wearden P, et al. Remote ischemic preconditioning decreases adhesion and selectively modifies functional responses of human neutrophils. J Surg Res. 2010;158:155–161. og Hess Hoda og Bhatia} En øgede udskillelse af endogen opioider kan have en betydning på aktivering og regulering af immunceller. Især den reducerede celledød ved behandling med RIC kan forbindes med øget immunrespons. Hormonel påvirker er også påvist i forbindelse med RIC. Den iskæmiske tilstand i kroppen har vist at øget udskillelsen af fx. adenosine og bradykinin. Begge stoffer har indvirkning på circulation og blodflowet. Bradykinin dilaterer blodkarrene og sænker dermed trykket og adenosine er flow regulerende, påvirker ATP produktion og medfører øget signal transmission over cellemembranen \fixme{Reference Hess, Hoda, Bhatia}

\begin{figure}[H]
	\centering
	\includegraphics[width = 0.7\textwidth]{billeder/Konditioneringsmekanismer.pdf}
	\caption{Oversigt over mulige mekanismer der kan aktiverings under konditionering} \label{fig:mechanism}
\end{figure}

\fixme{Billede til Hess, Hoda og Bhatia}
\subsection{Studieprotokol}

Mange faktorer omkring RIC er endnu ukendt og især effekten af behandling mangler evidence. Et kommende studie for neurologisk afsnit på Aarhus Universitetshospital (AUH) ønsker at undersøge effekten af RIPerC og RIPpostC, for at forbedre den kliniske rutine til behandling af patienter med akute iskæmiske stroke. \fixme{Rerence til studieprotokol} Studie skal undersøges på patienter med akut iskæmiske stroke(AIS), som kan modtage trombolysebehandling og patienter udvælges tilfældig så nogle patienter ikke vil modtage RIC. Studiet evaluerer patienter på række kriterier, heriblandt størrelsen af infarktet efter trombolyse, RIPerC og RIpostC og det kliniske output vurderet på \textit{modified Rankin Scale}. Der findes allerede studie, som har testet effekten af RIC på patienter med blodprop i hjerte og da disse ofte har et lavt blodtryk, findes der kun apparatet til okkludere armen ved 200mmHg. Da studie undersøger patienter med AIS, som kan have blodtryk på over 200mmHg skal studie bruge et modificeret blodtryksapparat som kan håndterer variationerne i blodtrykket og skabe tilstrække okklusion.

\section{Noninvasiv blodtryksmåling}\label{noninvasivBloodpressureMeasurement}
Noninvasiv blodtryksmåling, eller indirekte måling af det arterielle blodtryk er fællesbetegnelsen, for flere typer af tekniker, som alle estimerer blodtrykket i arteriet. Ofte associeres en blodtryksmåling af denne type, med den manuelle auditive detektion af puls, distal til en okkluderende manchet, som kan ses på figur \ref{fig:audiotoryBloodpressureMeasurement}. Denne manuelle auskulatoriske metode med kviksølvs sphygmomanometer anses stadig for at være guldstandarden inden for noninvasiv blodtryksmonitorering.\fixme{Requirements for professional office blood pressure monitors}

\begin{figure}[H]
	\includegraphics[width=0.9\textwidth]{billeder/TypicalIndirectBlood-pressureMeasurement.png}
	\caption{Typisk indirekte blodtryksmåling med sphygmomanometer, manchet ogstetoskop}\label{fig:audiotoryBloodpressureMeasurement}
\end{figure}
\fixme{ref: Webster side 325}

Det automatiske blodtryks apparat som erstatter den manuelle auditive metode (automatiseret auskultatorisk apparat) anvender i alt sin simpelhed en mikrofon i stedet for stetoskopet. Ultralyd anvendes også i nogle blodtryksapparater som erstatning af stetoskoppet og bestemmer ved hjælp af doppler, hvornår arteriet er total okkluderet af manchetten. Ultralyd har særlige fordele, såsom at kunne bruges på spædbørn og hypotensive patienter, hvor lyden af blodflowvibrationerne i arteriet kan være svære at hører. Langt de fleste blodtryksmållere anvender dog i dag den oscillometriske metode, hvor selve manchetten selv agerer som interface til det pulserende arterie (se figur \ref{fig:OscillometriskMetode}).\fixme{Requirements for professional office blood pressure monitors} Det ekspanderende arterie skubber til manchetten og skaber oscilloerende trykændringer i manchetten. På samme måde, som ved den auskultatoriske metode pumpes trykket i manchetten til over systolisk blodtryk, hvor arteriet er total okkluderet og manchetten udsættes på dette stadie ikke for pulsationer fra det underlæggene arterie. Luften i manchetten lukkes gradvist ud over tid. Når arterie trykket overstiger manchet trykket løber blodet ind i arteriet og skubber til arterievæggen. De små oscillotioner overføres til manchetten, hvilket resulterer i trykændringer (de største trykændringer i manchetten kan også observeres i sphygmomanometeret under en auskulatorisk måling). Oscillotionerne isoleres fra manchetrykket og kan ses på figur \ref{fig:OscillometriskMetode}. MAP ses hvor oscillotionerne er størst og det systoliske blodtryk ses hvor en pludseligt stigning i amplitude højden finder sted. Diastolen har ikke en klar overgang og er derfor bestemt ud fra algoritmer.\fixme{Webster side 328}

\begin{figure}[H]
	\includegraphics[width=0.9\textwidth]{billeder/OscillometriskMetode.png}
	\caption{Den oscillometriske metode. En kompressionsmanchet oppustes til et tryk over det systolisk blodtryk. Luften lukkes langsomt ud, hvorefter det systoliske tryk måles ved punkt 1 og MAP ved punkt 2. Det systoliske tryk ses ved den pluslige stigning i de oscillostionernes ampletuder og MAP er manchettrykket ved de største oscillotioner er til stede.}\label{fig:OscillometriskMetode}
\end{figure}
\fixme{ref: Webster side 329}

\section{Okklusionstræning}
Okklusionstræning eller blood flow resistance (BFR) træning har i det senest år gennemgået mange undersøgelse og har vist en stor effekt i forbindelse med muskel hypertrofi og styrke. Ved normal styrketræning skal en utrænede person arbejde omkring 45-60\% af 1 repetition maks (RM) for at opnå hypertrofi og øget styrke, og hos en trænet person skal man ligge omkring 80-85\% af 1-RM. Ved okklusionstræning skal belastningen ligge væsentlig lavere, omkring 20-50\% af 1-RM, for at opnå samme eller større effekt. 
Okklusiontræning udfører ved at afklemme blodforsyningen til muskel, så manchetten sidder proximalt for musklen. Trykket der okkluderes ved varierer meget fra studierne. Imens musklen er okkluderet arbejder personen til udtrættelse. Dette gentager i et ønskede antal set. 
Pga. det lave belastning og det relativt korte træning periode og stadig store effekt, egner det træningsform sig ideal for person med ledskader, til genoptræningsforløb eller til person som har været sengeliggende længe. \fixme{Reference til The efficacy of blood flow restricted exercise: A systematic review and meta-analysis}









