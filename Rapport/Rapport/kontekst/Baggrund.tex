\chapter{Baggrund} \label{chap:Baggrund}

Apopleksi (pludseligt opstået fokale neurologiske symptomer) opstår af infarkt eller en blødning. Ved infarkt nedsættes eller afbrydes blodforsyningen i visse område af hjernen og dette medfører iltmangel i det ramte område. I 85\% af tilfælde er apopleksi forårsaget af infarkt og 15\% skyldes blødning \fixme{Reference til "Basis i sygdomslære, side 399-402}

Hvert år indlægges ca. 12.000 danskere i forbindelse med apopleksi og i den vestlige verden er apopleksi det tredjehyppigste dødsårsag.\fixme{Reference program apopleksi, side 14}. Af de personer der overlever et apopleksi tilfælde, lever næsten 50\% af dem med varige men og 25\% af dem har behov for andres hjælp ved daglige aktiviteter. \fixme{Refence til fakta om apopleksi http://www.hjernesagen.dk/om-hjerneskader/bloedning-eller-blodprop-i-hjernen/fakta-om-apopleksi } Det høje antal tilfælde årligt og de mange personer med varige men har store omkostninger for sundhedssektoren.  I 2001 kostede apopleksi sundhedsvæsnet 1,8 milliarder kroner. \fixme{Reference til trombolyse økonomi side 17}

Den nuværende behandling af apopleksi og dets følgevirkning sker i flere forskellige trin; forbyggende, akut behandling og rehabilitering. 

Meget af den forebyggende behandling af apopleksi ligger i livstilsændringen. Faktorer for udvikling af apopleksi er bl.a. hypertension, hjerte-kar sygdomme, arteriosklerose og forhøjet kolesterol. 

For at opnå størst effekt af akut behandling af apopleksi skal behandlingen helst ske inden for 5 timer efter tilfældet indtræf. Behandlingen består som regel af en scanning for afgøre om der er tale om en blodprop eller en blødning. Hvis der er tale om en blodprop, vil patienten modtage trombolysebehandling 

Afhængig af méngraden består rehabiliteringen af genoptræning i forskellige form. Menene af apopleksi kan være alt fra talebesvær til halvsidig lammelse og derfor afhænger genoptræning også deraf. \fixme{https://www.sundhed.dk/borger/sygdomme-a-aa/hjerte-og-blodkar/sygdomme/apopleksi/behandling-ved-apopleksi/}

\section{Konditionering}









