\chapter{Metoder}
Metode kapitlet beskriver projektets arbejdsmetoder, hvilke metoder der brugt og hvordan de er blevet brugt. Metode vil især beskrive projektstyringsforløb og udviklingsmetoderne. 

\section{Projektstyring}
Til overordnede projektstyring er der gjort brug af den \textit{Struktureret Agile Metode}, forkortet SAM. (Se hjemmeside \fixme{Reference til http://www.agilemanifesto.org/iso/dk/}). Metoden karakteriseres ved at inddele projektet i følgende faser: krav, design, implementering og test. Metoden passede godt på projektet i flere omstændigheder. SAM er oplagt til projektgrupper i størrelsen 2-3 personer og projekter der var 3-9 måneder. Metoden er også særlig anvendelig til projektet da inddragelse af kunden fylder en stor del i arbejdet. 
Især i arbejde med forprojektet og opstartsfasen på projektet blev der afholdt mange møder for at fastlægge projektets rammer og kravene til produktet. I SAM metoden adskiller man møder i forskellige kategorier og de 3 kategori er som følger: \textit{introduktionsmøde, planlægningsmøde og kontraktmøde}. Samarbejdet med kunde Rolf Blauenfeldt kan meget vel inddeles i 3 forskellige møde kategorier. I forprojektet afholdte projektgruppen \textit{introduktionsmøde} med kunden for at forventningsafstemme. Da det var på plads og projektgruppen havde besluttet at kundens problemstilling var en opgave som gruppen kunne løse, blev der afholdt flere \textit{planlægningsmøder} for at finde og udspecificerer de  krav som kunden havde til produktet. Disse møder er afholdt over flere omgange, da der undervej i projekt er opstået situation, som ikke var blevet fastlagte. Men efter der var afholdt tilstrækkeligt \textit{planlægningsmøder} igangsatte projektgruppen første fase af SAM metode og der blev udarbejdet en kravspecifikation(Se afsnit \ref{title:kravspecifikation}). SAM metoder er et iterativ så undervejs i forløbet er der foretaget ændring og justeringer i kravspecifikationen. Kort efterfulgt af kravspecifikation er der udarbejdet en accepttest \ref{title:accepttest}, som bliver udfyldt når udviklingen af prototypen er færdigt. Inden arbejdet med prototypen begyndte, blev der udarbejdet et system design\ref{title:systemdesign}, for at fastlægge hvordan systemet skulle struktureres. 

\subsection{Scrum/Pivotaltracker}
Til arbejdsfordeling og planlægning af arbejdsopgaver er projektet udarbejde med hjælp af scrum. Der er ikke brugt scrum i direkte forstand. Men hver uge er blevet set som en sprint, hvor der hver mandag er udarbejdet en sprint backlog som skulle udføres i ugens løb. Emnerne til sprint backlogen er bla. taget fra tidsplanen som kan ses som en overordnet projekt backlog. Sidst på ugen er der afholdt møde, hvor der opsamles på ugens arbejdet og hvilke opgaver i sprint backlogen der er blevet løst. Opgaver, der ikke blev løst, er automatisk blevet videreført til næste uges backlog. Hver mandag når der oprettes et sprint backlog er disse opgaver blevet oprettet i projektstyringsværktøjet \textit{pivotaltracker}, se hjemmeside ()\fixme{https://www.pivotaltracker.com/}). 

\subsection{Samarbejdsaftale}
\subsection{Samarbejdspartnere}
\subsubsection{Kunde}
\subsubsection{Vejleder}
\subsubsection{Reviewgruppe}
\subsubsection{Eksperter}
\subsubsection{Firma}

\subsection{Logbog}
\subsection{Vejldermøde}
\subsection{Tidsplan}
\subsection{Tavshedspligt}

\section{Versionsstyring}

\section{Udviklingsværktøjer}

\section{Udviklingsproces}
	\subsection{Kravspecifikation} \label{title:kravspecifikation}
	
	\subsection{Accepttest} \label{title:accepttest}
	
	\subsection{System design} \label{title:systemdesign}
	
	\subsection{Implemetering} \label{title:implementering}

\subsection{V-model}



\subsection{Review}
