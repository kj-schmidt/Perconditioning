\chapter{Metoder}
Metode kapitlet beskriver projektets arbejdsmetoder, hvilke metoder der brugt og hvordan de er blevet brugt. Metodeafsnittet vil især beskrive projektstyringsforløb og udviklingsmetoderne. 

\section{Projektstyring} \label{title:projektstyring}
Til overordnede projektstyring er der gjort brug af den \textit{Struktureret Agile Metode}, forkortet SAM. (Se hjemmeside \fixme{Reference til http://www.agilemanifesto.org/iso/dk/}). Metoden karakteriseres ved at inddele projektet i følgende faser: krav, design, implementering og test. Metoden passede godt på projektet i flere omstændigheder. SAM er oplagt til projektgrupper i størrelsen 2-3 personer og projekter der varer 3-9 måneder. Metoden er også særlig anvendelig til projektet da inddragelse af kunden fylder en stor del i arbejdet. 
Især i arbejde med forprojektet og opstartsfasen på projektet blev der afholdt mange møder for at fastlægge projektets rammer og kravene til produktet. I SAM metoden adskiller man møder i forskellige kategorier og de tre kategori er som følger: \textit{introduktionsmøde, planlægningsmøde og kontraktmøde}. Samarbejdet med kunde Rolf Blauenfeldt kan meget vel inddeles i tre forskellige møde kategorier. I forprojektet afholdte projektgruppen \textit{introduktionsmøde} med kunden for at forventningsafstemme. Da det var på plads og projektgruppen havde besluttet at kundens problemstilling var en opgave som gruppen kunne løse, blev der afholdt flere \textit{planlægningsmøder} for at finde og udspecificerer de  krav som kunden havde til produktet. Disse møder er afholdt over flere omgange, da der undervej i projekt er opstået situation, som ikke var blevet fastlagte. Men efter der var afholdt tilstrækkeligt \textit{planlægningsmøder} igangsatte projektgruppen første fase af SAM metoden og der blev udarbejdet en kravspecifikation(Se afsnit \ref{title:kravspecifikation}). SAM faserne er iterative processer, derfor blev der undervejs i forløbet  foretaget ændring og justeringer i kravspecifikationen. Kort efterfulgt af kravspecifikation er der udarbejdet en accepttest \ref{title:accepttest}, som bliver udfyldt når udviklingen af prototypen er færdigt. Inden arbejdet med prototypen begyndte, blev der udarbejdet et system design \ref{title:systemdesign}, for at fastlægge hvordan systemet skulle struktureres. Da system strukturen var fastlagt begyndt implementeringsfasen og slutter med at gennemføre accepttesten. For uddybende information se afsnit \ref{title:udviklingMetode} omkring udviklingsdokumentationen.

\subsection{Scrum/Pivotaltracker} \label{title:scrum}
Til arbejdsfordeling og planlægning af arbejdsopgaver er projektet udarbejde med hjælp af scrum. Der er ikke brugt scrum i direkte forstand. Men hver uge er blevet set som en sprint, hvor der hver mandag er udarbejdet en sprint backlog som skulle udføres i ugens løb. Emnerne til sprint backlogen er bla. taget fra tidsplanen som kan ses som en overordnet projekt backlog. Sidst på ugen er der afholdt møde, hvor der opsamles på ugens arbejdet og hvilke opgaver i sprint backlogen der er blevet løst. Opgaver, der ikke blev løst, er automatisk blevet videreført til næste uges backlog. Hver mandag når der oprettes et sprint backlog er disse opgaver blevet oprettet i projektstyringsværktøjet \textit{pivotaltracker},( se hjemmeside \fixme{https://www.pivotaltracker.com/}). 

\subsection{Samarbejdsaftale}
For at sikre interne forventninger til projektarbejde i gruppen, har gruppens medlemmer lavet og underskrevet en samarbejdsaftale i begyndelse af projektet. \fixme{reference til samarbejdsaftale}

I forbindelse med samarbejdet med reviewgruppen er der også blevet udarbejdet og underskrevet en samarbejdsaftale, for at sikre ens forventning til reviewmøderne. \fixme{reference til samarbejdsaftale med review gruppen}

\subsection{Samarbejdspartnere}
Dette afsnit beskriver projektgruppens samarbejdspartnere igennem projektet. 

\textbf{Kunden og projektudbyder:} Rolf Ankerlund Blauenfeldt er læge ved neurologisk afsnit på Aarhus Universitet (AUH). Samarbejdet med Rolf har primært bestået i specificering af krav til udvikling af \textit{Konditioneringsapparatet} samt faglig ekspert for remote ischemic conditioning (RIC). Desuden det kunden som godkender accepttesten.

\textbf{Vejleder:} Projektvejleder Peter Johansen, har været som faglig vejleder i gennem hele projekt og igennem vejledermøder Peter bistået med faglig kritik løbende. 

\textbf{Reviewgruppe:} Igennem projektet har projektgruppen samarbejdet med en anden projektgruppen, Anders Esager og Anders Toft. Denne gruppe har fungeret som opponent/review gruppe, og hver gang en deadline var nået, fx accepttest, har grupperne reviewet hinanden opgaver, hvorefter et møde er blevet afholdet og rettelserne er blevet gennemgået.  

\textbf{Firma:} Virksomheden Seagul forhandler blodtryksapparatet og har i projektets opstart fungeret som kontaktperson til en kinesisk udviklingsvirksomheden. Det samarbejdet blevet oprettet for at projektgruppen kunne modtage teknisk sparring i udviklingsfasen. 

\textbf{Advokat:} I forbindelse med tavshedspligt (Se afsnit \ref{title:tavshedspligt}) har projektgruppen samarbejdet med juridisk rådgiver Maibrit Lerche Hendriksen fra Aarhus Universitet. Pga. projektgruppen ønskede samarbejde med en reviewgruppe, kunne samarbejdet ikke begynde før reviewgruppen også blev underlagt tavshedspligt 

\subsubsection{Samarbejde med medikoteknisk afd. AUH}
Til udvikling og kalibrering af \textit{Konditioneringsapparatet} har projektgruppen samarbejde med medikotekniske ingeniører  Sara Rose Newell og Steven Brantlov fra Region Midtjylland. Disse har kun bistå med blodtrykssimulator, samt teknisk forståelse af blodtryksmåling. Samarbejdet har bestået i mail korrespondance, samt to møder på medikoteknisk afsnit på AUH, hvor projektgruppen har testet og kalibreret \textit{Konditioneringsapparatet} på blodtrykssimulatoren. 

\subsubsection{Samarbejde med Troels Johansen}
I forbindelse med udvikling af sikkerhedskontrol til konditioneringsapparatet(Se afsnit \fixme{Reference til projektafgrænsning} omkring projektafgrænsninger) har projektgruppen samarbejdet med Troels Johansen fra lungeafdelingen på AUH. Samarbejdet opstod pga. gruppen manglede ekspertviden omkring pulsoximeteri og afklemning 

\subsection{Ugeplan og logbog}
Som del af projektstyringen, udviklingsprocess samt dagbog er der på ugentlig basis udarbejdet en ugeplan i starten af hver uge og hver uge er afsluttet med en logbog. Ugeplanen indholder de opgaver projektgruppen skal løses i ugens løb og logbogen er en opsamling på ugens arbejde. For uden af fungere som sprint backlog i scrum (Se afsnit \ref{title:scrum}) har logbogen også fungeret som en slags dagbog, hvordan projekts forløb konstant er blevet beskrevet. Logbogen har også været særlig anvendeligt forbindelse med rapport skrivning. 

\subsection{Vejldermøde}
Fra projektets opstart blev der aftalt et vejledermøde i alle ulige uger under projektforløbet. Disse møder er blevet brugt til at sikre at projektarbejdet hele tiden var på rette spor, samt faglig vejledning til projektarbejdet. Desuden er vejledermøderne blevet brugt til at få kritik på færdig dokumenter undervej i forløbet. 

\subsection{Tidsplan}
I forbindelse med forprojektet blev der udarbejdet en tidsplan i gantt chart format. Et gantt chart illutreret start og slut dato for hvert af projektet delelementer. Hver række i tidsplanen udgør et delelement, fx. kravspecifikation og accepttesten og hver kolonne udgør én uge. Tidsplanen er løbende blevet opdateret efterhånden som projekt har nået delelementerne.  \fixme{Insæt sidste opdateret tidsplan}

\subsection{Tavshedspligt} \label{title:tavshedspligt}
Pga. af patentundersøgelse har hele projekt været underlagt tavshedspligt og underskrevet tavshedserklæringer med både universitet og neurologisk afsnit. Tavshedspligten har bla. forsinket nogle processer da alle partner skulle være underforstået med fortroligheden inden et samarbejdet kunne begynde. I andre tilfælde hvor et samarbejde har været kortvarig eller der ikke har været til at underskrive tavshedserklæring, har projektgruppen måtte undlade detaljer ved kommunikation med disse samarbejdspartnere. Dette har i nogle tilfælde betyder at hjælpen fra evt. eksperter har været begrænset af manglende forståelse for projektet. Derfor har den igangværende patentundersøgelse været et begrænsning for projektarbejdet i flere omfang. 

\section{Versionsstyring}
For at sikre korrekt og brugervenlig versionsstyring af hele projektets versionsstyring blevet håndteret med git \fixme{Reference til git hjemmeside https://github.com/}. Git er versionsstyring primært udviklet til software. Styringen af versionshistorik fungere ved at man opretter et respository, som ligger på en server, og hver gang man ønsker at arbejde på filer en ens repository, skal man synkroniseret så man har senest version liggende. Foretages en ændring i en fil der køres versionshistorik på, skal denne ændres \textit{committes} til ens repository. Hver gang man tilføjer en ændring, skal man skrive hvilken ændring man har foretaget. Resten af versionsstyring foregår automatik i git, og her gemmes automatisk versionsnummer og dato for ændring. Desuden gør git det også nemt at gå tilbage i versionshistorikken og finde tidligere versionen. 
Selvom prototypen \textit{Konditioneringsapparatet} ikke skal medicinsk godkendes, var det en grund til at vælge et detaljeret versionsstyrings system. Hvis et apparat skal medicinsk godkendelse skal der kunne fremvises en versionshistorik over hele projektet. 
Til grafisk interface findes en række program som gør git og versionsstyring mere brugervenlig, og her har det især været brugbart at kunne se forrige ændringer og tilføjelse. Dette har lette projektarbejdet og mindsket uoverensstemmelser med hvilket dokument er nyeste version. 

Udover git er dropbox blev brugt til at dele projektfiler der ikke har behov for versionsstyring. Dette har fx. været videnskabelige artikler, datablade mm. 

\section{Udviklingsværktøjer}
\textbf{Eclipse:}

\textbf{Arduino IDE}

\textbf{Matlab:}

\textit{Fritzing:}

\textbf{Gimp:} 

\textbf{Maple:} Til filter udregning i forbindelse med design af analoge og digitale filtre er det blevet brugt Maple version 2015. Maple er kommercielt computer algebra system.  \\

\textbf{Modelio:}

\textbf{TexStudio:} 


\section{Udviklingsproces} \label{title:udviklingMetode}
Som et led af struktureret agil metodes(SAM) fire faser: krav, design, implementering og test (se afsnit \ref{title:projektstyring}) er udviklingsprocessen foregået efter disse faser. 

	\subsection{Kravspecifikation} \label{title:kravspecifikation}
	
	\subsection{Accepttest} \label{title:accepttest}
	
	\subsection{System design} \label{title:systemdesign}
	
	\subsection{Implemetering} \label{title:implementering}

\subsection{V-model}



\subsection{Review}
