\chapter*{Definitioner og forkortelser}
	\begin{longtable}{ |p{0.3\textwidth} |p{0.7\textwidth}| } 
		\hline
		\textbf{Udtryk / Forkortelse} &  \textbf{Forklaring} \\
		\hline
		RIC & Remote ischemic pre/per/post conditioning. Længerevarende okklusion af ydre ekstremitet, efterfulgt af en deflations fase\\
		\hline
		RIPreC, RIPerC og RIPostC & Hhv. remote ischemic preconditioning, remote ischemic perconditioning og remote ischemic postconditioning\\
		\hline
		AIS / apopleksi & Acute ischemic stroke, en pludseligt opstået neurologisk skade eller udfald på baggrund af iskæmi (nedsat blodforsyning) i hjernen \\
		\hline
		AUH & Aarhus Universitetshospital \\
		\hline
		\textit{Konditioneringsapparatet} & Navnet på prototype, som er udviklet til at udføre RIC \\
		\hline
		\textit{Okklusionsfase} & Periode hvor manchetten skaber arteriel okklusion \\
		\hline
		\textit{Deflationsfase} & Periode der altid er efterfulgt en okklusionsfase, hvor manchetten er deflateret i under 10mmHg\\
		\hline
		\textit{Cyklus} & Forløb bestående af én \textit{okklusionfase} og én \textit{deflationsfase} \\
		\hline
		\textit{Gennemført afklemning} & Boolean værdi der bruges til at bestemme om en cyklus er gennemført eller ej \\
		\hline
		\textit{Tid pr. cyklus} & Den tid en \textit{okklusions-} eller \textit{reperfusionsfase} tager at gennemføre. \\
		\hline
		\textit{Antal cyklusser} & Antallet af cyklusser en RIC behandling varer \\
		\hline
		SYS, SP & Forkortelse for systolisk tryk \\
		\hline
		DIA, DP & Forkortelse for diastolisk tryk \\
		\hline
		MAP & Middel arterie trykket \\
		\hline
		\textit{Konditioneringsapparat} & Navnet på det modificerede blodtryksapparat og prototypen som projektgruppen har udviklet \\
		\hline
		Repository & Lagring plads, hvorpå projektets filer versionsstyres \\
		\hline
		Auskulatorisk metode & Metode hvorpå der lyttes efter korotkoff lyde ved afklemning af et arterie \\
		\hline
		Sphygmonanometer & Analog trykmåler med mmHg som enhed, brugt til blodtryksmåling  \\
		\hline
		Shield & Et tilføjelsesprint, som udvider funktionaliteten af arduino'en \\
		\hline
		SysML & Open source visuelt modelleringssprog, til forklaring og analyse af specifikation, design mm.  \\
		\hline
		UML & Et standard modelleringssprog til visualisering af diagrammer for softwaredesign \\
		\hline
		IDE & Integreret udviklingsmiljø \\
		\hline
		
		
	\end{longtable}
\newpage