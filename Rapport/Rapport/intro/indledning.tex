\chapter{Indledning}
Neurologisk afdeling på Aarhus Universitetshospital skal i et kommende studie undersøge effekten af \textit{Remote Ischemic Pre- og Postcondtioning}. \textit{Remote ischemic conditiong (RIC)} er en behandlingsform, som har vist lovende resultater i forebyggelse og behandling af patienter med blodprop i hjertet. Studiet ønsker at undersøge, om denne behandlingsform kan have samme gavnlige effekt på patienter med apopleksi. For at studiet skal kunne gennemføres, har afdelingen kontaktet Ingeniørhøjskolen, Aarhus Universitet, omkring udvikling af det apparat, som skal udføre RIC behandlingen under studiet. Denne udviklingsopgave blev til dette bachelorprojekt. 

\section{Formål}
Formålet med denne projektrapport er at beskrive projektforløbet i forbindelse med udviklingen af \textit{Konditioneringsapparatet}. Rapporten beskriver den faglige baggrund for projektet og behovet for et modificeret blodtryksapparat (konditioneringsapparat). Hernæst gennemgås hvilke afgrænsninger, projektgruppen har foretaget. Ydermere er formålet med rapporten at sikre, læseren har en forståelse for, hvilke metoder der er anvendt i projektforløbet samt resultatet og diskussion af den opnåede prototype. 

\section{Læsevejledning}
Projektrapporten skal læses som resultatet af projektforløbet. Rapporten er opdelt i følgende kapitler;

	\begin{longtable}{ p{0.14\textwidth} p{0.8\textwidth} } 
		Kapitel 1 & Præsentation af indledende punkter omkring bachelorprojektet, \textit{Remote ischemic conditioning}\\
		Kapitel 2 & Her beskrives den viden, der ligger til grund for forståelsen af projektet. \\
		Kapitel 3 & Beskrivelse af problemformuleringen, som er udarbejdet i samarbejde med kunden og vejlederen\\
		Kapitel 4 & Præsentation af områder, hvor projektet er blevet afgrænset\\
		Kapitel 5 & Giver en kort beskrivelse af den udviklede prototype\\
		Kapitel 6 & Redegørelse for hvilke ingeniørfaglige metoder projektet har gjort brug af \\
		Kapitel 7 & Præsentation af de centrale opnåede resultater ved produktudviklingen og projektforløbet\\
		Kapitel 8 & Beskrivelse og gennemgang af diskussionspunkter omkring de centrale resultater\\
		Kapitel 9 & Redegørelse for fremtiden for prototypen\\ 
		Kapitel 10& Indeholder en opsamling og konklusion på projektforløbet og prototypen\\
	\end{longtable}

\subsubsection{Appendiks}
Appendiksafsnittet indeholder logbog, tidsplan, samarbejdsaftale mm. Dette afsnit skal læses som dokumentation for projektforløbet

\subsubsection{Udviklingsdokumentation}
Foruden projektsrapporten består projektgruppens skriftlige produkt også af et udviklingsdokument. Dette udviklingsdokument giver læseren fuldt indblik i udviklingsfasen af \textit{Konditioneringsapparatet}. Dokument består af 4 underdokumenter, hhv. kravspecifikation, accepttest, system design og implementering. 

Dokumentationsrapporten bygger på udviklingsdokumentationen, som af denne grund er fungerende som opslagsdokument for dokumentationsrapporten. Se udviklingsdokumentet i vedlagte ZIP-fil under \textit{Konditioneringsapparat/udviklingsdokument.pdf}.



