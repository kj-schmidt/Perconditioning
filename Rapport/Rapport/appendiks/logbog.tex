\section{Ugeplan og logbog} \label{title:ugeplanOgLogbog}

	\subsection{Uge 36}
	\begin{longtable}{|p{0.24\linewidth}|p{0.7\linewidth}|}
		\hline
		Uge nr.: & 36 (31/8-6/9) \\ \hline
		Ugeplan & 
		\begin{itemize}
			\item Valg af værktøjer:
			\begin{itemize}
				\item Projektstyring
				\item Versionsstyring
				\item Kildestyring 
				\item Latex
			\end{itemize}
			\item Planlægning af møder
			\begin{itemize}
				\item Vejlder 
				\item Review gruppe
			\end{itemize}
			\item Komponent bestilling
			\item Udformelse af skabeloner til møder mm. 
			\item Tidsplan
			\item Kravspec
		\end{itemize}
		
		\\ \hline
		Logbog & 
		\begin{itemize}
		\item Projektsstyring
			\begin{itemize}
			\item Valg af værktøjer 
				\begin{itemize}
				\item Latex til rapport og Design dokumentation
				\item Google Docs til logbog, tidsplan, møder mm.
				\end{itemize}
			\item Arrangeret review møder med review gruppe (Anders Toft og Anders Esager)
			\item Arrangeret vejleder møder med Peter Johansen
			\item Anvender Pivotaltracker og har i den sammenhæng oprettet deadlines for projektet samt milestones.
			\end{itemize}
		\item Indkøb og lån af komponenter:
			\begin{itemize}
			\item Pumpe, ventil, cuff, arduino UNO og motorshield
			\end{itemize}
		\item Opdatering af tidsplan
		\item Kravspecifikation - opdatering
		\item Oprettelse af skabeloner for Logbog, reviewmøder, ugeplan og vejledermøder 
		\item Kildestyring
			\begin{itemize}
		\item Oprettelse af Refwork bruger og en række kilder
			\end{itemize}
		\end{itemize}
		\\ \hline
	\end{longtable}
	
	\subsection{Uge 37}
	\begin{longtable}{|p{0.24\linewidth}|p{0.7\linewidth}|}
		\hline
		Uge nr.: & 37 (7/9-13/9) \\ \hline
		Ugeplan & 
		\begin{itemize}
			\item Kravspec
			\begin{itemize}
				\item Use cases
				\item System arkitektur
				\item 	SysML
			\end{itemize}
			\item Kildelæsning
			\begin{itemize}
				\item 	Perkonditionering
				\item Oximetri 
				\item Litteratursøgning 
			\end{itemize}
		\end{itemize}
		
		\\ \hline
		Logbog & 
		\begin{itemize}
			\item Opdatering af kravspecifikation
			\begin{itemize}
				\item Fully dressed use case diagrammer
				\begin{itemize}
					\item Tilføjelse af 2 nye use cases og opdatering af de forrige
				\end{itemize}
				\item Ikke funktionelle krav
				\item SysML
				\begin{itemize}
					\item Use case diagram 
					\item Sequence diagram
				\end{itemize}
				\item Illustrationer 
			\end{itemize}
			\item Vejleder møde med Peter d. 7/9
			\item Møde med Rolf d. 10/9
			\item Valg af versionsstyringsværktøj
			\begin{itemize}
			\item 	Git / GitHub
			\end{itemize}
			\item Valg af SysML / UML værktøj
			\begin{itemize}
				\item Modelio 
			\end{itemize}
		\end{itemize}
		\\ \hline
	\end{longtable}
	
	\subsection{Uge 38}
	\begin{longtable}{|p{0.24\linewidth}|p{0.7\linewidth}|}
		\hline
		Uge nr.: & 38 (14/9-20/9)\\ \hline
		Ugeplan & 
		\begin{itemize}
			\item Kravspec
			\begin{itemize}
				\item Ikke funktionelle krav
				\item SysML
			\end{itemize}
			\item Bestil ved Farnell
			\begin{itemize}
				\item MPXV5100GC6V
				\item 2x 165-0687
				\item Display
			\end{itemize}
			\item Aflevér kravspec og accepttest til review gruppe
			\item Github
			\item Accepttest
		\end{itemize}
		\\ \hline
		Logbog & 
		\begin{itemize}
				\item Kravspec
				\begin{itemize}
					\item Ikke funktionelle krav
					\item SysML
				\end{itemize}
				\item Bestilt ved Farnell
				\begin{itemize}
					\item MPXV5100GC6V
					\item 2x 165-0687
					\item Display
				\end{itemize}
				\item Fortrolighedsaftale med reviewgruppe
				\item Opsætning af arduino og strømforsyning.
				\item Modtaget komponenter
				\item Aflevér kravspec og accepttest til review gruppe
				\item Github
				\item Accepttest
		\end{itemize}
		\\ \hline
	\end{longtable}
	
	\subsection{Uge 39}
	\begin{longtable}{|p{0.24\linewidth}|p{0.7\linewidth}|}
		\hline
		Uge nr.: & 39 (21/9-27/9)\\ \hline
		Ugeplan & 
		\begin{itemize}
			\item Vejledermøde 
			\item Skaf “slangemuffer”
			\item Gennemarbejde af review materiale 
			\item Status over dokumentation
			\item Versionsstyring af source kode
			\item Pivotaltracker skal opdateres
			\item Systembeskrivelse og illustration 
			\item Prototypemål
			\begin{itemize}
				\item Inflatere (hastighed) 
				\item Deflatere cuff(tidsstyring)
				\item Opsætning af de forskellige “programmer
				\item Intern hukommelse
			\end{itemize}
		\end{itemize}
	
		\\ \hline
		Logbog & 
		\begin{itemize}
			\item Vejledermøde 
			\item Fundet en løsning på samling af slangerne - afventer svar fra Rasmus
			\item Gennemarbejde af review materiale 
			\item Status over dokumentation
			\begin{itemize}
				\item Opdateret kravspec og accepttest til ny skabelon og tilføjet indledning
				\item Næste skridt: System arkitektur
			\end{itemize}
			\item Versionsstyring af source kode
			\begin{itemize}
				\item Køres over Git, ligger under /Prototype
			\end{itemize}
			\item Pivotaltracker skal opdateres
			\begin{itemize}
				\item Skal tjekkes om morgen og inden man går hjem
			\end{itemize}
			\item Systembeskrivelse og illustration 
			\begin{itemize}
				\item Systembeskrivelse er færdigt. Mangler oversigtstegning
			\end{itemize}
			\item Prototypen 
			\begin{itemize}
				\item Udskiftning af arduino 
				\item Styre ventilen
				\item Hastighedsstyring af motoren
			\end{itemize}
		\end{itemize}
		\\ \hline
	\end{longtable}
	
	\subsection{Uge 40}
	\begin{longtable}{|p{0.24\linewidth}|p{0.7\linewidth}|}
		\hline
		Uge nr.: & 40 (28/9-4/10)\\ \hline
		Ugeplan & 
		\begin{itemize}
			\item Følg op joint tupes fra Rasmus
			\item Systemarkitektur 
			\begin{itemize}
				\item UC1- UC8
			\end{itemize}
			\item Kravspec og accepttest
			\begin{itemize}
				\item Ret sysML
			\end{itemize}
			\item Møde med Troels
			\item Evt. mødes Stefan Wagner
			\begin{itemize}
				\item Kig på pulsoximeter
			\end{itemize}
		\end{itemize}
		
		\\ \hline
		Logbog & 
		\begin{itemize}
			\item System Architecture
			\begin{itemize}
				\item State machine diagrams
				\item IBD (Hardware)
				\item Overordnet beskrivelse af systemets dele
				\item Beskrivelse af 4 + 1 modellen
				\item Beskrivelse af BDD og domæne model
			\end{itemize}
			\item Indledende undersøgelser omkring arduinoens DAQ egenskaber
			\begin{itemize}
				\item Sampling rate
				\item ADC bits
				\item Regnekraft
				\item RAM (til lokale variabler)
			\end{itemize}
			\item Møde med Troels ang. pulsoximeter
			\begin{itemize}
				\item NIRS er vores bedste mulig pga af bl.a tiden og videnskabelig evidens
			\end{itemize}
		\end{itemize}
		\\ \hline
	\end{longtable}
	
	\subsection{Uge 41}
	\begin{longtable}{|p{0.24\linewidth}|p{0.7\linewidth}|}
		\hline
		Uge nr.: & 41 (5/10-11/10)\\ \hline
		Ugeplan & 
		\begin{itemize}
			\item Systemarkitektur
			\begin{itemize} 
				\item Implementeringsview
				\begin{itemize}
					\item IBD 
				\end{itemize}
			\end{itemize}
			\item Projektafgrænsninger
			\begin{itemize}
				\item Arduino, RAM
				\item 10-bit ADC 
				\item Processorkraft 
			\end{itemize}
			\item Designdokumentation
		\end{itemize}
		
		\\ \hline
		Logbog & 
		\begin{itemize}
			\item Systemarkitektur 
			\begin{itemize}
				\item Processview
				\begin{itemize}
					\item State machines for all senarier
				\end{itemize}
				\item Implementationview 
				\begin{itemize}
					\item IBD hardware
					\item Class diagram software
					\item Udviklingsvæktøjer
				\end{itemize}
				\item Deployment view
				\begin{itemize}
					\item Beskrivende tekst
				\end{itemize}
				\item Google docs overført til LaTex
				\begin{itemize}
					\item Alt dokumentation er overført til Latex 
				\end{itemize}
			\end{itemize}
			\item Projektafgrænsninger
			\begin{itemize}
				\item Arduino, RAM
				\item 10-bit ADC 
				\item Processorkraft 
				\item Endnu ikke dokumenteret 
			\end{itemize}
			\item Dokumentation
			\begin{itemize}
				\item Rettet billede- og tabel test for accepttest, kravspec og system arkitektur
				\item Styring af float objekter i latex - brug aldrig pagebreak 
			\end{itemize}
			\item Komponent bestilling
			\begin{itemize}
				\item Fitting til ventilslange og sensor er skaffet
				\item Fitting til pumpe og manchet samt t-rør er bestilt hos Rasmus - kommer i løbet af næste uge 
			\end{itemize}
		\end{itemize}
		\\ \hline
	\end{longtable}
	
	\subsection{Uge 42}
	\begin{longtable}{|p{0.24\linewidth}|p{0.7\linewidth}|}
		\hline
		Uge nr.: & 42 (12/10-18/10)\\ \hline
		Ugeplan & 
		\begin{itemize}
			\item Systemarkitektur 
			\begin{itemize}
				\item Gøres klar til review
			\end{itemize}
			\item Opfølgning på komponenter(joint tupes) 
			\item Anskaf reference pulsoximeter
			\item Test sensor
		\end{itemize}
		
		\\ \hline
		Logbog & 
		\begin{itemize}
			\item Systemarkitektur 
			\begin{itemize}
				\item Gjort klar til review og rettet igen af SVG
				\begin{itemize}
					\item Indsættelse af figurtekster 
				\end{itemize}
			\end{itemize}
			\item Joint tupes er leveret og udkast til “systemopsætning” er lavet
			\item Tryksensor
			\begin{itemize}
				\item Målt tryk med sensoren og det er blevet sammenholdt med det analog sphygmonanometer
				\item Tætning af kredsløb
			\end{itemize}
			\item Skaffet reference pulsoximeter
			\begin{itemize}
				\item Lavet kort test hvor puls og sat observeres efter 5 mins afklemning - umiddelbart ikke noget brugbart
			\end{itemize}
			\item Test sensor
		\end{itemize}
		\\ \hline
	\end{longtable}
	
	\subsection{Uge 43}
	\begin{longtable}{|p{0.24\linewidth}|p{0.7\linewidth}|}
		\hline
		Uge nr.: & 43 (19/10-25/10)\\ \hline
		Ugeplan & 
		\begin{itemize}
			\item Review af systemarkitektur 
			\item Kalibrering af sensor
			\item Dataopsamling 
			\item Tætning af kredsløb
			\item Opsætning af software efter klassediagram
			\begin{itemize}
				\item Namespaces og dokumentstruktur
			\end{itemize}
			\item Opfølgning på display 
			\item Test setup til at måle oscillationer 
		\end{itemize}
		
		\\ \hline
		Logbog & 
		\begin{itemize}
			\item Review af systemarkitektur 
			\begin{itemize}
				\item Læst og rettet review gruppens system Ark og afholdt review møde. 
			\end{itemize}
			\item Kalibrering af sensor
			\begin{itemize}
				\item Fik fjernet start off-set
				\item Tilføjede buffer - sensor kan ikke sende stor nok spændingen til arduino pga af lav indgangsimpedens 
			\end{itemize}
			\item Dataopsamling 
			\begin{itemize}
				\item Design af 2. ordens butterworth lav pas filter
				\item Design af 1. ordens høj pas filter til fjernelse af DC
				\item Design af forstærkning kredsløb
			\end{itemize}
			\item Tætning af kredsløb
			\begin{itemize}
				\item Påsætning af tætningstape(PTFE tape)
				\item Systemet har stadig et lille leak, men under 10mmHG pr 30 sek
			\end{itemize}
			\item Opfølgning på display 
			\begin{itemize}
				\item Displayet er stadig under levering, men midlertidig display er lånt med samme opløsning og udvikling af grænseflade er i gang. 
				\begin{itemize}
					\item Displayet kan skifte mellem de 3 programmer; konditionering, okklusion og setup
					\item Setup programme kører delvist
				\end{itemize}
			\end{itemize}
			\item Test setup til at måle oscillationer 
			\begin{itemize}
				\item Har snakket med Sara Rose Newell og vi kan muligvis låne fantom setup til at måle oscillationer
			\end{itemize}
		\end{itemize}
		\\ \hline
	\end{longtable}
	
	\subsection{Uge 44}
	\begin{longtable}{|p{0.24\linewidth}|p{0.7\linewidth}|}
		\hline
		Uge nr.: & 44 (26/10-1/11)\\ \hline
		Ugeplan & 
		\begin{itemize}
			\item Software implementering
			\begin{itemize}
				\item Klasser
				\begin{itemize}
					\item PressureControl
					\item MotorControl
					\item Sensoring
				\end{itemize}
				\item Dokumentering af implementering
			\end{itemize}
		\end{itemize}
		
		\\ \hline
		Logbog & 
		\begin{itemize}
			\item Software implementering
			\begin{itemize}
				\item Displayet
				\begin{itemize}
					\item Setup er færdig, kan skifte mellem værdier og ændre værdier
					\item Occlusion er færdig, kan starte og stoppe opdatering af sensor værdi og timer
					\item Conditioning kan start og stoppe behandlingsforløb med timer og sensor aflæsning
					\begin{itemize}
						\item Mangler implementering af blodtryksknap
					\end{itemize}
				\end{itemize}
				\item Filter design
				\begin{itemize}
					\item 2. ordens butterworth ved cut off på 11Hz
					\item Høj pas filter knækker 0.2 Hz
					\item Gain på x100
				\end{itemize}
				\item Signalbehandling på arduino
				\begin{itemize}
					\item Problematik omkring mangel på ram - delvist løst
					\item Detektering af toppunkter 
				\end{itemize}
				\item Monitorering af strøm belastning fra motoren
			\end{itemize}
			\item Review
			\begin{itemize}
				\item Aftalt deadline for implementeringsdokument d. 7. nov og review d. 11. november
			\end{itemize}
			\item Fantomtest
			\begin{itemize}
				\item Torsdag d. 3. nov har vi lånt fanton test setup til test af manchet oscillationer.  
			\end{itemize}
			\item Komponentbestilling
			\begin{itemize}
				\item Bestilling af modeswitch 
				\item Display ankommer i denne uge
			\end{itemize}
		\end{itemize}
		\\ \hline
	\end{longtable}
	
	\subsection{Uge 45}
	\begin{longtable}{|p{0.24\linewidth}|p{0.7\linewidth}|}
		\hline
		Uge nr.: & 45 (2/11-8/11)\\ \hline
		Ugeplan & 
		\begin{itemize}
			\item Prototype skal kunne måle blodtryk til den 5/11, hvor aparatet skal måle på et fantom.
			\begin{itemize}
				\item Printet skal lodes op
				\item Sofwaren skal kunne måle et MAP og et SYS som minimum og gerne kunne måle DIA også.
			\end{itemize}
			\item Yderligere implementering af software
			\begin{itemize}
				\item Software til at holde styr på tiden og fremvise resterende cyklus tid skal vises på display.
				\item Metode til at gemme Setup indstillinger i EEPROM’en
			\end{itemize}
			\item Implementering dokument skal skrives til review deadline
		\end{itemize}
		
		\\ \hline
		Logbog & 
		\begin{itemize}
			\item Software implementering
			\begin{itemize}
				\item Namespace
				\begin{itemize}
					\item GUI
					\begin{itemize}
						\item Display og Buttons er implemeteret 
					\end{itemize}
					\item Logic
					\begin{itemize}
						\item Der er indkøbt en real time clock, til præcis styring af tiden og til at indhente tidsstempler
						\item RTC’en er udskifter timeren der bruger interrupt og muliggør delays i koden
						\item Impl. af midlingsfilter 
						\item Ændre af samplingsteknik, så der tjekkes over et vindue med 13 samples efter peaks. 
					\end{itemize}
					\item Data 
					\begin{itemize}
						\item Der kan nu læses og skrives fra SD kortet
						\item EEPROMs adresserne 100 og 101 er allokeret til at gemme tid pr cyklus og antal cyklusser
					\end{itemize}
				\end{itemize}
			\end{itemize}
			\item Hardware implementering 
			\begin{itemize}
				\item Alle filtre, knapper og sensor er blevet samlet på et print, som passer oven på arduinoen lige som et shield. Undervejs i ugen var der store problemer med støj på signal. Og testen på medico tekniske afd. fejlede da der var for meget støj på signalet. Vi har derfor aftalt ny tid hvor vi kan få lov til at teste igen. Grunden til støjen var input benene på en schmitt trigger der ikke var “tøjret”. 
				\item Vi venter stadig på display shieldet. 
			\end{itemize}
		\end{itemize}
		\\ \hline
	\end{longtable}
	
	\subsection{Uge 46}
	\begin{longtable}{|p{0.24\linewidth}|p{0.7\linewidth}|}
		\hline
		Uge nr.: & 46 (9/11-15/11)\\ \hline
		Ugeplan & 
		\begin{itemize}
			\item Dokumentation af software afleveres onsdag d. 12/9 og der laves review fredag. 
			\item Prototypen skal være “færdig” fredag d. 13/9 og softwaren skal være samlet i et projekt
			\item Møde med Rolf ang. kravspec og accepttest d. 10/9 
			\item Møde med Kristian ang. occlusionstræning d. 11/9
			\item Test på fantomarm d. 13/9
		\end{itemize}
		
		\\ \hline
		Logbog & 
		\begin{itemize}
			\item Dokumentation af prototype, det vil sige implementeringsdokumentet blev udskudt til fredag den 13/11. hvilket flyttede reviewet til mandag den 16.
			\item Prototypen er bygget sammen med display og knapper. Der mangler stadig en forening af softwaren.
			\item Møde med Rolf ang. kravspec og accepttest d. 10/9 gik godt. han var meget positiv indstillet.
			\item Møde med Kristian ang. occlusionstræning d. 11/9. Se møde refferat
			\item Test på fantomarm d. 13/9 er udført på trods af mange indledende problemer med simulatoren, som ikke fungerer med ventil i kredsløbet.
		\end{itemize}
		\\ \hline
	\end{longtable}
	
	\subsection{Uge 47}
	\begin{longtable}{|p{0.24\linewidth}|p{0.7\linewidth}|}
		\hline
		Uge nr.: & 47 (16/11-22/11)\\ \hline
		Ugeplan & 
		\begin{itemize}
			\item Implementeringsdokument
			\begin{itemize}
				\item Tilføj beskrivelse af motorshield, timer og display
				\item Tilføj opfyldt krav. 
				\item Review
			\end{itemize}
			\item Opdatér tidsplan
			\item Samle kode 
			\item Vejledermøde
			\item Fordel opgaver til rapporten. 
		\end{itemize}
		
		\\ \hline
		Logbog & 
		\begin{itemize}
			\item Reviewmøde 
			\begin{itemize}
				\item Stor forskel på AT og AE implementeringsdokument.
				\item Enige om at målgruppen skal specificeres i læsevejledning
				\item Planlagt af vi reviewer rapporten i så bider
				\item Udarbejdet skabelon til rapport 
			\end{itemize}
			\item Implementeringsdokument
			\begin{itemize}
				\item Tilføjet beskrivelse af motor shield, knapper og display
				\item Fikset rettelser efter review møde
			\end{itemize}
			\item Opdatér tidsplan
			\item Samle kode 
			\begin{itemize}
				\item Der var en del komplikation ved samlingsprocessen, og nogle ting kunne godt have været gjort anderledes
				\item RTC er loddet på prototypen
				\item Prototypen mangler nu kun at få ændre få threshold værdier for puls detektion og at få monteret en modeswitch
				\item Prototypen er færdig i denne uge. 
				\begin{itemize}
					\item Foruden strømforsyning
				\end{itemize}
			\end{itemize}
			\item Vejledermøde
			\begin{itemize}
				\item PJO ønskede filter afsnit i rapport, hvor der beskrives resultatet af filteringen
			\end{itemize}
		\end{itemize}
		\\ \hline
	\end{longtable}
	
	\subsection{Uge 48}
	\begin{longtable}{|p{0.24\linewidth}|p{0.7\linewidth}|}
		\hline
		Uge nr.: & 48 (23/11-29/11)\\ \hline
		Ugeplan & 
		\begin{itemize}
			\item Rapport
			\begin{itemize}
				\item Baggrundsafsnit 
				\item Problemformulering
				\item Resultatafsnit
				\begin{itemize}
					\item Ratiosfikseret metode 
					\item Signal behandling(filtrering)
					\item Bruger interface
					\item Data loging 
				\end{itemize}
				\item Projektstyring
			\end{itemize}
			\item Prototype
			\begin{itemize}
				\item Strømforsyning
				\item Ret koden igennem(kommentarer, ubrugte variabler etc.) 
				\item Ret ventil under okklusionstræning
			\end{itemize}
			\item General prøve af accepttest
		\end{itemize}
		
		\\ \hline
		Logbog & 
		\begin{itemize}
			\item Rapport
			\begin{itemize}
				\item Baggrundsafsnit er færdigt, skal reviewes
				\item Problemformulering er færdig, skal til review
				\item Resultatafsnit indeholder nu
				\begin{itemize}
					\item Ratiosfikseret metode 
					\item Signal behandling(filtrering)
					\item Data loging 
				\end{itemize}
				\item Metode afsnittet er udarbejdet, skal nu rettes igennem
				\item Projektafgrænsninger indeholder nu: 
				\begin{itemize}
					\item Sikkerhedskontrol 
					\item MR kompatibilitet 
					\item Samarbejdet med Seagull
				\end{itemize}
				\item Diskussionsafsnit:
				\begin{itemize}
					\item BT med fikseret ratio. 
				\end{itemize}
			\end{itemize}
			\item Prototypen
			\begin{itemize}
				\item Ventil fungere nu optimalt under okklusionstræning
			\end{itemize}
			\item General prøve af accepttest
			\begin{itemize}
				\item D. 26/11 blev der gennemført generalprøve på accepttesten. 
				\item Der blev foretaget små rettelser, og tilpasninger i koden og kravene.
			\end{itemize}
		\end{itemize}
		\\ \hline
	\end{longtable}
	
	\subsection{Uge 49}
	\begin{longtable}{|p{0.24\linewidth}|p{0.7\linewidth}|}
		\hline
		Uge nr.: & 49 (30/11-6/12)\\ \hline
		Ugeplan & 
		\begin{itemize}
			\item Accepttesten 
			\begin{itemize}
				\item Afholdes d. 30/11
			\end{itemize}
			\item Review af del 1 af 2 af rapporten
			\begin{itemize}
				\item Afholdes d. 1/12
			\end{itemize}
			\item Rapport
			\begin{itemize}
				\item Systembeskrivelse 
				\item Resultater
				\item Diskussion
			\end{itemize}
			\item Projektdokumentation
			\begin{itemize}
				\item Samle KS, AT, system design og implementering 
			\end{itemize}
		\end{itemize}
		
		\\ \hline
		Logbog & 
		\begin{itemize}
			\item Accepttesten 
			\begin{itemize}
				\item Gennemført d. 30/11. Resultatet blev at alle use cases blev godkendt, med undtagelse af UC5 (sikkerhedskontrol). Desuden blev ét punkt i UC7 delvist godkendt, pga den ikke stoppede med det samme. Den manglende godkendelse af UC5 var forventet, da den ikke var implementeret
			\end{itemize}
			\item Review af del 1 af 2 af rapporten
			\begin{itemize}
				\item Der blev afholdt review møde d. 1/12 omkring første del af rapport. Der blev primært rettelse konjunktur fejl, men ellers blev det nævn at der manglede figurer i metode afsnit. Det var også et ønske fra review gruppen at der skulle være flere resultater af selve prototypens funktion
			\end{itemize}
			\item Færdige afsnit i rapport, som mangler konjunktur læsning
			\begin{itemize}
				\item Baggrund, Problemformulering, projektafgrænsning, metodeafsnit, resultater, perspektivering
				\item Alle bilag er lagt i latex og ligger nu i appendiks i rapporten. Appendiks er logbog, mødereferater, samarbejdsaftale, tidsplan, puls test, kalibreringstest og tavshedserklæring
			\end{itemize}
			\item Projektdokumentation
			\begin{itemize}
				\item Alt udviklingsdokumentation er samlet i ét dokument med samme navn. Hver underdokument har sin egen indholdsfortegnelse og kan stadig læses separat. 
			\end{itemize}
			\item Vejledermøde
			\begin{itemize}
				\item Der blev afholdt vejledermøde d. 3/12 med PJO. Se mødereferat for uge 49. 
			\end{itemize}
		\end{itemize}
		\\ \hline
	\end{longtable}
	
	\subsection{Uge 50}
	\begin{longtable}{|p{0.24\linewidth}|p{0.7\linewidth}|}
		\hline
		Uge nr.: & 50 (7/12-13/11)\\ \hline
		Ugeplan & 
		\begin{itemize}
			\item test \fixme{skal udfyldes}
		\end{itemize}
		
		\\ \hline
		Logbog & 
		\begin{itemize}
			\item test
		\end{itemize}
		\\ \hline
	\end{longtable}
	

	
	