\chapter*{Resume}
\textbf{Baggrund}
\textit{Remote ischemic conditioning(RIC)} er en behandlingsform, som har vist sig effektiv til behandling og forebyggelse af iskæmiske skader i hjertet. Nu vil et forskningshold fra Aarhus Universitetshospital undersøge om samme effekt kan opnås på patienter med apopleksi. Til udførsel af RIC behandlingen skal forskningsholdet have udviklet et apparat, der kan måle blodtrykket hos apopleksi patienter og derefter påbegynde behandlingen. RIC behandlingen foregår ved at okkludere en ekstremitet med en blodtryksmanchet ved et tryk på minimum 25 mmHg højere end systolisk tryk. Behandlingen udføres i cyklusser, hvor en cyklus består af én okklusionsfase og én reperfusionsfase. 

\textbf{Metoder}
Ved hjælp af agile og iterative processer er der blevet udarbejdet en prototype, som kan udfører RIC behandling. Udviklingsfasen betod af fire faser, hhv. kravspecifikation, system design, detaljeret design og implementering. For at sikre en konstant fremgang i projektforløbet er udviklingsfasen blevet styret ved hjælp af SCRUM. I projektet er der kørt ugentlig sprint og projektets backlog har været en tidsplan udarbejdet som et gantt chart. Sammen med tidsplanen er milepæle i projektet nået ved hjælp af en review gruppe. I samarbejde med reviewgruppen blev der fastlagt deadlines for hvornår hver milepæl skulle være færdig.

\textbf{Resultater og diskussion}
Accepttesten af prototypen viste at kravene til produktet var delvist opfyldt. Prototypen kan måle et blodtryk, hvor efter trykket i manchetten fyldes til 25mmHg over det målte systoliske tryk. Herefter påbegyndes RIC behandlingen. Da prototypen skal bruges til et forskningsprojekt gemmes information omkring RIC behandlingen på et SD kort, når prototypen udfører en behandling. Det er også muligt at ændre det antal cyklusser en behandling skal vare, samt den tiden pr cyklus. Dette resultat er implementeret hvis studieprotokollen på et senere tidspunkt ønsker at ændre opsætningen for RIC behandlingen. Foruden RIC behandlingen kan prototypen også indstilles til at udføre okklusionstræning. Grunden til at accepttesten kun var delvist godkendt, var at prototypen ikke er implementeret med en sikkerhedskontrol der kan afbryde RIC behandlingen. 

\textbf{Konklusion}
Da projektgruppen foruden den, fra starten, projektafgrænsede sikkerhedskontrol under RIC behandling, blev projektets mål opnået. Den nye viden omkring behovet for et modificeret blodtryksapparat blev omsat til en funktionel prototype, der kunne udføre den ønskede RIC behandling. Ydermere blev implementeret en ekstra funktion i form af okklusionstræning. De ingeniørfaglige metoder sikrede at udviklingsfasen forløb flydende og at deadlines blev overholdt. Mulighederne med RIC og prototypen kan på lang sigt have stor betydning for både patienter og sundhedssektoren.