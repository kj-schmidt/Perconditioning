\chapter{Konklusion}
I dette projekt blev der udarbejdet en prototype, til konditionering af apopleksi patienter, specialdesignet til et forskningsprojekt af RIC på Aarhus Universitetshospital. Udviklingen af apparatet til konditionering er sket i samarbejde med flere eksperter, (læger, ingeniører, fysiologer) samt en gennemgang af litteraturen inden for området. Prototypen, kaldet \textit{Konditioneringsapparatet}, er også udviklet til brug i fremtidige studier inden for okklusionstræning.
Den udviklede prototype (se figur \ref{fig:oversigtsbillede}) giver mulighed for at vælge imellem konditionering eller okklusionstræning, hvorefter apparatet selv regulerer trykket i manchetten efter behov. \textit{Konditioneringsapparatet} er ved hjælp af analog og digital signalbehandling, også i stand til at måle blodtrykket non-invasivt. Det målte blodtryk anvendes i konditioneringsbehandlingen til korrekt afkleming af armen. \textit{Konditioneringsapparatet} logger selv data fra konditioneringen, hvilket gør den yderst anvendelig til forskning. Funktionen sikkerhedskontrol er ikke blevet implementeret, fordi pulsoximetri viste sig ikke at være anvendeligt til dette.
Udviklingen af konditioneringsapparatet blev styret gennem udarbejdelse af kravspecifikation, system design og implementering med metoderne SAM, SCRUM og V-model. Kvalitetssikring af udviklingen er sket gennem versionsstyring og en dedikeret review gruppe.
Fremtidig videreudvikling af \textit{Konditioneringsapparatet}, bør indeholde overvejelser omkring implementering af en anden oscillometisk analyse metode og medicinsk godkendelse, som vil åbne op for flere fremtidige studier.
\textit{Konditioneringsapparatet} vil, hvis forskningsprojektet af RIC giver gode resultater, give mulighed for en bedre behandling af AIS på vej til sygehuset, på sygehuset og efterfølgende i hjemmet. Dette giver anledning til en økonomisk gevinst for både patienten, regionen og kommunen. 

\label{SidsteSide}

\textbf{Læringsmål:}\fixme{skal ikke med i rapporten}
\begin{itemize}
	\item Omsætte forskningsresultater samt naturvidenskabelig, sundhedsvidenskabelig og teknisk viden til anvendelse ved udviklingsopgaver og ved løsning af sundhedsteknologiske problemstillinger
	\item Søge, analysere og vurdere ny viden indenfor relevante områder
	Udvikle nye løsninger
	\item Anvende ingeniørfaglig teori og metode på en systematisk måde
	\item Vurdere og forklare projektresultater for ingeniører og andre målgrupper, skriftlig og mundtligt
	
	\item Reflektere over anvendelsen af projektresultaterne i til relation til sociale, organisatoriske, arbejdsmiljømæssige, økonomiske og etiske konsekvenser
\end{itemize}




