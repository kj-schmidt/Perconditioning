\chapter*{Abstract}
\textbf{Background} \textit{Remote ischemic conditioning (RIC)} is an intervention, that has shown great improvement in preventions and treatment of ischemic heart disease. Department of Neurology at Aarhus University Hospital want to conduct a study on the effects of RIC on patients with acute ischemic stroke. For this study the department needs a special device capable of adjusting the RIC according to the patient's blood pressure. RIC is performed by creating ischemic occlusion on a limb. The occlusion is created with a cuff, and the pressure has to be 25 mmHg above the systolic blood pressure. RIC is performed in cycles consisting of one occlusion phase and one reperfusion phase. 

\textbf{Methods} The development of the prototype, capable of perform RIC treatment, is created using both agile and iterative processes. The development can be separated into four different phases respectively; system requirement, system design, detailed designed and implementation. To ensure a fluent project process, the development of the prototype was controlled with SCRUM. A sprint was lasting one week at the time, and the project backlog consisted of the elements displayed in the project schedule. The project schedule was created as a gantt chart. The schedule and a dedicated review group ensured that the product reached its milestone, because deadline for milestone was created in collaboration with the review group. 

\textbf{Results and discussion} The system verification showed that the requirement to the product where partially fulfilled. The prototype is capable of determinating the level of pressure in the cuff according to the blood pressure to ensure a total occlusion of the limb. Since the prototype is a device for a research project, all information about the RIC treatment is saved onto an SD card. With the prototype is it possible to change the number of cycles and the time per cycle for a RIC. The prototype can also perform blood flow restricted training. The reason why the system verification only was partially approvement is that the prototype is build without a \textit{safety control}. 

\textbf{Conclusion}
The project process the need of a modified blood pressure monitor into a working prototype, capable of performing RIC treatment. With exception of the \textit{safety control} the prototype fulfilled the required functions. The extra function for creating restricted blood flow during physical exercise was also implemented. The use of engineering methods ensured a fluent project procedure and that all deadlines were met. The future possibilities with RIC and the prototype could show great improvement for AIS treatment and the economy of the health sector. 