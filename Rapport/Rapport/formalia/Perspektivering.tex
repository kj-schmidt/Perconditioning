\chapter{Perspektivering} 

\section{Prototypen}
Da \textit{konditioneringsapparatet} kun er en prototype, kræves et stort stykke arbejde til udviklingsfasen endnu, før apparatet kan tages i brug i studiet eller til almen brug. Selve funktionalitet af apparatet er på plads, men størrelse, udseende, udeevnt, samt produktion er en række de område hvor \textit{konditioneringsapparatet} mangler udvikling. En oplagt retning at udvikle prototypen i, ville være at designet et print. Dette ville reducere størrelse af prototypen og gøre det nemmere at designe et housing til produktet. Arduino platformen hæmmer kompleksitet af print design fasen, da platformen er open source.  \fixme{Jeg synes der skal skrives mere, men jeg kan ikke finde på hvad det skal være}

\section{Sikkerhedskontrol}\label{title:nirs}
Selvom pulsoximetri viste sig ikke at være gyldig som sikkerhedskontrol, finders der alternativer, som også benytter optik til at monitorere kredsløbet. NIRS er en teknik, der i stil med pulsoximeteri også benytter optik i det nær infrarøde område af lysspektret, nemmere betegnet 700-1100 nm. NIRS står for \textit{Near InfraRed Spectroscopy} og adskiller sig fra pulsoximetri på flere parametre. Pulsoximetri kigger kun på absorption fra det pulserende blod i arterierne, hvor NIRS analyserer det optiske respons fra arterier, vener og kapillærer. Dette giver, i stedet for en saturation, et index for hvor godt vævet er iltet, kaldet \textit{Tissue Saturation Index} (TSI). Afhængig af hvilken NIRS teknologi man benytter, giver NIRS enten et mål for ændringer i iltkoncentration, eller den absolutte iltkoncentration i det belyste væv. Begge dele ville være særdeles egnede som sikkerhedskontrol til \textit{Konditioneringsapparatet}, da begge giver en indikation af hvor godt iltet vævet er, \cite{RefWorks:22} .

NIRS adskiller sig yderligere fra pulsoximetri ved at lyskilder og lysmodtager sidder på samme side, og kan derfor monteres direkte i forlængelse af manchetten. Her skal pulsoximetri sidde på fingeren og det sætter krav til enten kablet eller trådløs kommunikation med \textit{Konditioneringsapparatet}. 

Udover NIRS kunne fungere som en sikkerhedskontrol ved konditioneringsbehandling, så vil NIRS også være gavnlig ved okklusionstræning. Kunne man lave en manchet, hvor der på den distale side sad en NIRS sensor, ville man kunne monitore hvor hurtigt musklen brugte sin iltreserve. Hvis denne reserve når under en hvis threshold værdi, kunne man bruge det som en indikator for et godkendt okklusionstrænings set. Som træningsformen bruges i dag, udføres hver set til \textit{udtrættelse}. Dette er en subjektivt vurdering og vil kunne variere meget fra person til person, og fra dag til dag. Med NIRS monitorering sammen med okklusionstræning vil man sikre at musklen når til samme udtrættelse hver gang. 

\section{Kombinering af RIC og okklusionstræning} \label{title:kombRICogOkkl}
\textit{Remote ischemic conditioning} (RIC) laver midlertid iskæmi i det afklemte område. Ved belastning af en muskel øges dens ilt behov, og det intracellulære tryk stiger under hver kontraktion. Desto større belastning og længere tid musklen skal arbejde, desto mere iskæmisk bliver muskel. På den måde opstår en lignende tilstand, som ved behandling med RIC. Ved okklusionstræning opstår denne tilstand hurtigere, da iltforsyningen til musklen er begrænset. Denne samhørighed er belyst i litteraturen (\cite{RefWorks:3}), men især efter samarbejde med Kristian Vissing (Se afsnit \ref{title:samarbejdspartnere} omkring samarbejdspartnere), blev projektgruppen opmærksom på den tilsvarende tilstande. Kristian Vissing lagde op til at man kombinerede RIC behandling med okklusionstræning(Reference \fixme{møde med kristian}). Da \textit{Konditioneringsapparatet} både kan udfører okklusionstræning og RIC er det oplagt at videreudvikle disse funktionaliteter, så det kan sammenkobles. 

Patienter, der rammes af AIS, har stor risiko for at blive ramt igen, da de allerede har arteriosklerose. Endvidere risikere nogle AIS patienter at blive invalideret af sygdommen, og ender med at være sengeliggende i en periode efter tilfældet. I begge tilfælde vil der være behov for at forbygge risikoen for et nyt tilfælde af AIS. Men hvis patienten også har været sengeliggende efter, vil der også opstå et behov for genoptræning. Her er kombinationen af okklusionstræning og konditioneringsbehandling særdeles oplagt. Okklusionstræning medfører øget hypotrofi og RIC behandlingen forbygger et nyt AIS tilfælde.   \fixme{Er du enig Simon? Jeg kører med den store gyllespreder lige nu ;) }

Som beskrevet i afsnit \ref{title:nirs} omkring sikkerhedskontrol med NIRS, vil det være oplagt at implementere NIRS i en videreudvikling af \textit{Konditioneringsapparatet}. Med NIRS monitoreringen og kombination af okklusionstræning og RIC opstår også muligheden for at sammenligne effektten af de to funktioner. Hvis en person der udfører okklusionstræning opnår samme TSI værdi, som en person der modtager konditioneringsbehandling, kunne det være interessant at undersøge om okklusionstræning har tilsvarende effekt som RIC og okklusionstræning kan gøre det ud for RIC behandling. 

\section{Hjemmebehandling}\label{title:Hjemmebehandling}
Da RIC behandling både kan bruges som før, under og efterbehandling mod AIS, er det oplagt at \textit{Konditioneringsapparatet} kan sendes med patient efter udskrivelse. Som beskrevet i afsnit \ref{title:kombRICogOkkl} er det række patient grupper hvor både RIC behandling og okklusionstræning er særlig gavnligt. Set fra patient og sygehusets side er der store fordele ved at behandlingen kan fortsætte i hjemmet. Fra patientens side, betyder hjemmebehandling, kortere tid på sygehuset og for hver behandling der foretages mindskes risikoen for et nyt tilfælde. Fra sygehuset og sundhedssektorens side er hjemmebehandlingen en kæmpe økonomisk fordel, desto før patient kan udskrives, desto mindre udgift per patient. 
For at \textit{Konditioneringsapparatet} skal kunne sendes med patienten hjem, skal apparatet CE godkendes (Se afsnit \ref{title:medGodkendelse})