\chapter{Perspektivering} 

\section{Prototypen}
Da \textit{konditioneringsapparatet} er en \textit{proof of concept} prototype, kræves der et stykke arbejde til udviklingsfasen endnu, før apparatet kan tages i brug til RIC studiet. Funktionaliteten af apparatet er på plads, men størrelse, udseende, samt produktion er en række af de område hvor \textit{konditioneringsapparatet} mangler videreudvikling. En oplagt retning at udvikle prototypen i, vil være at designe et print til alle komponenter. Dette vil reducere størrelsen af prototypen og forsimple designfasen af et housing til produktet. Arduino platformen hæmmer kompleksiteten af print design fasen, da platformen er open source. I print design fasen ville det være oplagt at genbruge processoren på microcontroller, ATMEGA2560, og udlade resten af komponenter på Arduino Mega'en, for at mindske produktets fysiske størrelse. 

En medicinsk godkendelse af \textit{konditioneringsapparatet} (se afsnit \ref{title:medGodkendelse}) vil i fremtiden sikre et større anvendelsesområde af apparatet til forsøg og studier, men samtidig også åbne op for salg af apparatet og derved et økonomisk grundlag for masseproduktion.

\section{Sikkerhedskontrol}\label{title:nirs}
Selvom pulsoximetri viste sig ikke at være gyldig som sikkerhedskontrol, findes der alternativer, som også benytter optik til at monitorere iltningen af vævet. NIRS er en teknik, der i stil med pulsoximeteri også benytter optik i det nær infrarøde område af lysspektret, nemmere betegnet 700-1100 nm. NIRS står for \textit{Near InfraRed Spectroscopy} og adskiller sig fra pulsoximetri på flere parametre. Pulsoximetri kigger kun på absorption fra det pulserende blod i arterierne, hvor NIRS analyserer det optiske respons fra hele det belyste område, arterielt blod, venøst blod og væv (se figur \ref{fig:opticTissue}). Dette giver, i stedet for en saturation af kun det arterielle blod, et index for hvor godt vævet er iltet, kaldet \textit{Tissue Saturation Index} (TSI). Afhængig af hvilken NIRS teknologi man benytter, giver NIRS enten et mål for ændringer i iltkoncentration, eller den absolutte iltkoncentration i det belyste væv. Begge dele ville være særdeles egnede som sikkerhedskontrol til \textit{Konditioneringsapparatet}, da begge giver en indikation af hvor godt iltet vævet er, (\cite{RefWorks:22}) .

NIRS adskiller sig yderligere fra pulsoximetri ved at lyskilder og lysmodtager sidder på samme side, og kan derfor monteres direkte i forlængelse af f.eks. manchetten, hvor imod pulsoximetri skal sidde på fingeren og det sætter krav til enten en kablet eller trådløs kommunikation til \textit{Konditioneringsapparatet}. 

Udover at NIRS kunne fungere som en sikkerhedskontrol ved konditioneringsbehandling, så vil NIRS også være gavnlig ved okklusionstræning. Kunne man lave en manchet, hvor der på den distale side sad en NIRS sensor, ville den kunne monitore hvor hurtigt musklen brugte sin iltreserve. Hvis denne reserve når under en hvis threshold værdi, kunne man bruge det som en indikator for et godkendt okklusionstrænings set. Som træningsformen bruges i dag, udføres hvert sæt til \textit{udtrættelse}. Netop fordi \textit{udtrættelse} er en subjektivt vurdering og vil kunne variere meget fra person til person, og fra dag til dag, er det et problem. Med NIRS monitorering, sammen med okklusionstræning, er det muligt at sikre den samme udtrættelse af musklen hver gang. 

\section{Kombinering af RIC og okklusionstræning} \label{title:kombRICogOkkl}
\textit{Remote ischemic conditioning} (RIC) laver midlertidig iskæmi i det afklemte område. Denne tilstand opstår også under hård fysisk træning. Ved belastning af en muskel øges dens ilt behov, og trykket i musklen stiger under hver kontraktion. Desto større belastning og længere tid musklen skal arbejde, desto mere iskæmisk bliver muskel. På den måde opstår en lignende tilstand, som ved behandling med RIC. Ved okklusionstræning opstår den iskæmiske tilstand hurtigere end normal træning, da iltforsyningen til musklen er begrænset. Denne samhørighed er belyst i litteraturen (\cite{RefWorks:3}), men især efter samarbejde med Kristian Vissing (Se afsnit \ref{title:samarbejdspartnere} omkring samarbejdspartnere), blev projektgruppen opmærksom på den tilsvarende tilstand. Kristian Vissing lagde op til muligheden at kombinerede RIC behandling med okklusionstræning(Se mødereferat \ref{app:kristianuge46}). Da \textit{Konditioneringsapparatet} både kan udfører okklusionstræning og RIC er det oplagt at videreudvikle disse funktionaliteter, så de kan sammenkobles til én behandling. 

Patienter, der rammes af AIS, har stor risiko for at blive ramt igen. Endvidere risikerer nogle AIS patienter at blive invalideret af sygdommen, og ender med at være sengeliggende i en periode efter tilfældet. I begge tilfælde vil der være behov for at forbygge risikoen for et nyt tilfælde af AIS. Hvis patienten samtidig også har været sengeliggende, vil der også opstå et behov for genoptræning. Her er kombinationen af okklusionstræning og konditioneringsbehandling særdeles oplagt. Okklusionstræning medfører øget hypotrofi og RIC behandlingen forbygger et nyt AIS tilfælde. Forsøg/studier på en kombination af de to behandlinger bør overvejes, på grund af de store forbedringer af den allerede eksisterende behandling det kan have. 

Som beskrevet i afsnit \ref{title:nirs} omkring sikkerhedskontrol med NIRS, vil det være oplagt at implementere NIRS i en videreudvikling af \textit{Konditioneringsapparatet}. Med NIRS monitoreringen og kombination af okklusionstræning og RIC opstår også muligheden for at sammenligne effektten af de to behandlinger. Hvis en person der udfører okklusionstræning opnår samme TSI værdi, som en person der modtager konditioneringsbehandling, kunne det være interessant at undersøge om okklusionstræning har tilsvarende effekt som RIC og om okklusionstræning kan gøre det ud for RIC behandling. 

\section{Hjemmebehandling}\label{title:Hjemmebehandling}
Da RIC behandling både kan bruges som før, under og efterbehandling mod AIS, er det oplagt at \textit{Konditioneringsapparatet} kan sendes med en patient hjem efter udskrivelse. Som beskrevet i afsnit \ref{title:kombRICogOkkl} er der en række patientgrupper hvor både RIC behandling og okklusionstræning er særlig gavnligt. Set ud fra patientens og sygehusets synspunkt kan der store fordele ved at behandlingen kan fortsætte i hjemmet. For patienter betyder hjemmebehandling kortere tid på sygehuset, som mindsker belastningen på antallet af sengepladser og er dermed en mindre økonomisk byrde. Samtidig øges sikkerheden for patienten, fordi risikoen for et nyt slagtilfælde mindskes (\cite{RefWorks:20}). Fra sygehuset (regionens) og kommunens side er hjemmebehandlingen en økonomisk fordel, fordi det betyder færre sengepladser og den bedre behandling af patienten giver samtidig færre symptomer og følgesygdomme. 
For at \textit{Konditioneringsapparatet} skal kunne sendes med patienten hjem, skal apparatet CE godkendes (Se afsnit \ref{title:medGodkendelse})